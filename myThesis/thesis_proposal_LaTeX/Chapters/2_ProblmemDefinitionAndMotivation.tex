% Chapter Template

\chapter{Problem Definition and Motivation} % Main chapter title

\label{Chapter:ProblemDefinitionandMotivation}

%\subsubsection*{\color{mygray}[Chapter ready for review]}
% Describe the current situation of this problem.
% identifying the obstacles and the possible applications of your research.
% background of the problem (e.g. origins, previous attempts, etc.). 
% State the problem in detail mentioning all relative aspects, variables, relationships,
% stressing on the importance of finding a solution to the problem.

Carbon nanowires have been fabricated with a photoresist by multiple-photon polymerization techniques. However little is known about polymers that can produce conductive carbon nano-wires after pyrolysis, as it is generally believed that most polymers do not form significant amounts of graphitic carbon when carbonized.
%The lack of research relays on the fact that in the past years, it was assumed that most polymers are non-graphitic through pyrolysis \cite{Franklin1951}.
In the past years, photopolymerization processes have been applied to the fabrication of nano-structures with the use of an epoxy based photoresist. \cite{Boer2014} Photopolymerization techniques deliver patterning resolutions with nano-scale tolerances through two-photon lithography for the production of highly detailed structures \cite{Hribar2014}.

On the other hand, electrospinning has been acknowledged as a process with promising results at nano-structure fabrication \cite{Boer2014}, yet there is little research regarding the implementation of electrospinning for the fabrication of carbon nano-wires. Electrospinning has the potential to be a more straightforward process for the design and fabrication of nano-structures, as it can achieve mass scale manufacturing in a continuous, simple and reproducible manner. Cardenas \cite{Cardenas2017} showed that electrospinning can be implemented with ease for carbon nano-wire synthesis. Mechano-electrospinning, a new variant of electrospinning shows promising results in the production of ordered carbon nano-wires. As stated in \cite{Cardenas2017}, mechano-electrospinning is an early technology invention and brings new challenges, such as the reproducibility of carbon nano-wire production. Furthermore, the study of a new fabrication process to produce carbon nanowires that involves mechano-electrospinning will enable spatial control of the structures' patterning.

Since electrospinning seems to be a better alternative for carbon nano-wire fabrication processes; and for that purpose of its implementation, it is required to develop polymer solutions that can be mechano-electrospun, photopolymerized and pyrolyzed into conducting carbon nano-wires. Carbon nano-materials have been subjected to research due to their various potential applications in diverse areas that take advantage of the nano-scale properties. \cite{Siddiqui2019} Carbon nano-materials are suitable for the catalysis, adsorption, carbon capture, energy and hydrogen storage, drug delivery, bio-sensing and cancer detection. \cite{Siddiqui2019} However most applications are not currently feasible due to the lack of a continuous, simple and reproducible fabrication method with inexpensive processes. With the newly designed polymer solution, it would be possible to produce carbon nano-wires in large quantities, and therefore more applications will become feasible. On the other hand, the new technique will overcome some limitations of other methods such as lithography currently has. For instance, patterns created by lithography processes cannot be originated, only replicated, all constituent points of the pattern can only be addressed at the same time, and the process requires the pattern to be encoded into a mask. \cite{Landis2011}


%----------------------------------------------------------------------------------------
%	SECTION 1
%----------------------------------------------------------------------------------------



%-----------------------------------
%	SUBSECTION 1
%-----------------------------------

