% Chapter Template

\chapter{Carbon Nanowires Research Developments in Terms of Published Papers, Synthesis and Fabrication} % Main chapter title

\label{Chapter:2}

Nanotechnology ability to control or assemble materials at the nanoscale has fostered the development of diverse nanomaterials and nanostructures, including quantum dots [1], nanofibers, nanotubes [2] and nanowires [3,4]. These nanomaterials are of particular interest for biomedical applications in neurology, where conductive materials may promote electrical and chemical communication within the nervous system at the micro- and nano- scale levels. Applications of nanostructures to neuroscience have rapidly expanded from molecular imaging [5], to neuro- regenerative scaffolds [6] and neural interfaces [7-9]. 

In this framework, carbon-based nanomaterials (CBNs) and in particular nanotubes deserve particular attention, due to the exponential increase in neuroscience applications of materials composed mainly by carbon with different hybridization or struc- tures [10].

In this chapter we focus on the applications of CBNs-based technology to provide a picture of past and ongoing research in this field, highlighting the goals that have been achieved and the insights reached in understanding CBNs.

%----------------------------------------------------------------------------------------
%	SECTION 1
%----------------------------------------------------------------------------------------
\section{Carbon and carbon-based nanomaterials}

Carbon is the most versatile element in the periodic table [11], owing to the large number of bonds of different type and strength that can form with it or with many other elements. Moreover, the ability of carbon orbitals to hybridize in sp, sp2 and sp3 configura- tions paves the way to the existence of a number of allotropes. To date, the three naturally occurring allotropes of carbon (diamond, amorphous carbon and graphite), have been joined by additional ones deriving from synthetic processes (such as graphene, carbon nanotubes, fullerenes, carbon nanohorns, nanodiamonds) [ [12]; Fig. 1].

The interest in CBNs has increased exponentially in the last
decades, first with the discovery offullerenes (1985), then with that of carbon nanotubes (CNTs; 1991) and finally with the synthesis of graphene (GR) (2004).

The properties of these CBNs make them widely used in many
fields ranging from material science [13], energy production and storage [14], environmental sciences [15,16], biology [17-19] and medicine [20,21]. Table 1 summarizes the main properties of the most common CBNs [22-25]:

Among the many carbon nanomaterials, CNTs and GR are currently the most popular representatives and have been exten- sively studied for their excellent mechanical strength, electrical and thermal conductivity and optical properties. The Young's modulus and tensile strength of CNTs and GR can reach 1 TPa and 130 GPa respectively [20,21]. Carrier mobility of graphene is around 860 $cm^{2} V^{-1} s^{-1}$ (hole mobility of 844 $cm^{2} V^{-1} s^{-1}$ and carrier mobility of 866 $cm^{2} V^{-1} s^{-1}$), and the current density of metallic CNTs is orders of magnitude higher than those of metals such as copper [26,27]. Thermal conductivities of CNTs and GR are about 3000e3500W/mK and 5000 W/mK respectively [28]. The light absorption ratio of single-layer graphene is limited to 2.5\% [29]. A large amount of the research efforts were focused on exploiting these properties for various applications including electronics, biological engineering, filtration, lightweight/strong composite materials, photovoltaic and energy storage [30-32]. CNTs and GR are naturally good electrical conductors and their biocompatibility can be modulated [33], making them good candidates for improving electrodes for neural interfaces. Electrical recording or stimulation of nerve cells is widely employed in neural prostheses (for hearing, vision, and limb-movement recovery), in clinical therapies (treating Parkinson's disease, dystonia, and chronic pain), as well as in basic neuroscience studies. In all these applications, electrodes of various shapes and dimensions stimulate and/or re- cord neuronal activity to directly modulate behavior or to interface machine. The performance of the electrodes can be significantly improved by implementing the device with nanomaterial-based coatings (such as CBNs), since their high surface area can drastically increase charge injection capacity and decrease the interfacial impedance with neurons [34].

%%%
It is well known that the morphology of the electroactive material play an important role in electrochemical systems, once the intimate contact at the electrode/electrolyte interface is crucial to guarantee the charge transfer process [11,12]. In this sense, nanos- tructured electroactive materials such as nanorods [13], nanofiber [14], nanoflowers [15,16], nanowires [17,18], among others [19,20], are being widely used to obtain improved electrochemical properties such as higher charge/discharge rate and improved task to accommodate stress due to volume changes occasioned by intercalation process.

%----------------------------------------------------------------------------------------
%	SECTION 2
%----------------------------------------------------------------------------------------
\section{Carbon nanowires}

Carbon nanofibers (CNFs) have been classified as linear, sp2-based discontinuous filaments, where the aspect ratio is greater than 100 [207]. Depending on the angle of the graphene layers that compose the filament, CNFs have even been classified as stacked (graphene layers stacked perpendicularly to the fiber axis) or herringbone/cupstacked (graphene layers stacked at an angle be- tween parallel and perpendicular to the fiber axis) [208].

%%%
Nanowires are one-dimensional, anisotropic structures, small in diameter, and large in surface-to-volume ratio. These characteris- tics confer to the nanowires special physical properties than those of traditional scale and dimensionality materials, such as electrical, optical thermal and mechanical properties. However, this make this kind of materials to have properties deeply dependent on their surface condition and geometrical configuration [21]. The trans- port properties in the 1D nanostructures like the nanowires are affected by wire diameter, surface conditions, crystal quality, crystallographic orientation and material composition, thus the synthesis conditions are a crucial factor to obtain reproducible and high-quality nanowires for different application [21].

The typical lengths and diameters of carbon nanofibers are in the ranges of 5e100mm and 5e500 nm, respectively [209]. CNTs and GR are the most studied carbon nanomaterials for neural in- terfaces, however CNFs are also attractive in bio-interfacing de- velopments due to their chemical and physical properties [1]: CNFs are chemically stable and inert in physiological environment [2], they are biocompatible for long-term implantation due to CNFs solid carbon skeleton [3], they are electrically robust and conduc- tive for signal detection [4], they can be manufactured into 3D structures allowing intra-tissue and intracellular penetration [210], CNFs possess high surface-to-volume ratio, which greatly reduces electrical impedance, and [5] ultra-micro scale sizes that provide high spatial resolution. CNFs have been applied as promising ma- terials in many fields, such as energy conversion and storage, reinforcement of composites and self-sensing devices.

In addition, CNF based materials have been developed as electroconductive scaffolds for neural tissues to facilitate communication through neural interfaces. Electrical fields are able to enhance and direct nerve growth [211], therefore electroconductive scaf- folds have been applied to enhance the nerve regeneration process, not only providing physical support for cell growth but also delivering the functional stimulus. CNFs may represent novel, versatile neural interfaces, being capable of dual-mode operation by detecting electrophysiological and neurochemical signals, not only at the extracellular level with high spatial resolution, but also at the intracellular level by penetrating into single neurons [9].

Despite the longstanding experience on these nanomaterials and the deep knowledge of the CNFs-neuron interface in vitro, in vivo experiments on their possible application for the treatment of brain and spinal cord injuries or diseases are still limited to few examples [53,212,213]. In the first report CNFs impregnated with subventricular stem cells were employed to promote neuro- regeneration after experimental stroke [53]. The animals receiving the CNF-based treatment show reduction of the infarcted volume as well as recovery ofmotor and somatosensory activity. These data indicate that CNFs are optimal support material for neuronal tissue regeneration [53].

Recently, Guo and collaborators [104] developed a polymer-based neural probe with CNFs composites as recording electrodes via the thermal drawing process [213]. They demonstrated that in situ CNFs alignment was achieved during the thermal drawing, which contributes to a drastic improvement of electrical conductivity by 2 orders of magnitude compared to a conventional polymer electrode. The resulting neural probe has a miniature footprint, with a recording site reduced in size to match single neuron, yet maintaining impedance value able to capture neural signals. In chronic settings, long-term reliable electrophysiological recordings with single-spike resolution and minimal tissue response over extended period of implantation in wild-type mice were shown [213].

%%%
Nanostructured materials are particularly good for supercapacitor applications, providing high surface area, which leads to a high specific capacitance [22]. Compared to 3D and 2D materials, 1D nanostructures have smaller dimension and higher aspect ratio, improving the transport of electrical carriers in one controllable direction and also can be exploited as elements in different kinds of nanodevices [23]. In this way, nanowires have been satisfacto- rily used in supercapacitor electrodes due to their reduced ion dif- fusion path in comparison with 2D and 3D nanostructures, leading in higher charge/discharge rates [22,24].

%%%
Amongst the innumerous materials used to obtain 1D nanostructures, such as, carbon, silicon, transition metal oxides, the 1D nanostructured conductive polymers are a important group to fabricate energy storage devices, due their attractive characteristic, such as, mechanical properties, electrical conductivity, low cost, easy processing, high surface area and unique electroactive behavior, including high voltage window and high-doping rate during charge-discharge process [25]. During the charge/discharge process in the conducting polymer occur the insertion/desertion ions from the electrolyte in the polymer backbone that could result in swelling and shrinkage of the polymer chain, leading to mechanical degradation of the electrodes and fading the electrochemical performance [26]. An alternative to diminishment this drawback of the conductive polymers is fabricate composites that could improve the stability and conductivity of the electrodes [27].

%%%
In this way, composites based on conducting polymers and carbon or metal oxides materials in a nanowire architecture are a good strategy to develop high-performance devices due to the combination of the electrochemical properties of the polymer and/or composites with the morphological advantages of the nano- wires. These combination results in large interface between electrode/electrolyte, effective electronic transport pathway, short ion diffusion distance and easy relaxation strain, which could improve both capacity/capacitance and rate performance of battery and supercapacitors devices, respectively [28]. Furthermore, the mechanical properties of nanowires allow the development of flexible devices, that require materials with versatile functional- ities including high flexibility and foldability without losing its high power and energy density and long lifetime [29].

%----------------------------------------------------------------------------------------
%	SECTION 3
%----------------------------------------------------------------------------------------
\section{Nanowires synthesis}

%%%
Numerous methods to prepare nanowires, which include template-assisted synthesis [30], vapor–liquid-solid (VLS) [31], electrodeposition [32], electrospinning [33], hydrothermal [34], also hierarchical arrangement techniques [35–37] to organize the nanowires have been studies in the last ten years. Nanowires based on organic, inorganic or hybrid materials have been applied in order to get single or composites nanomaterials for innumerous purposes, such as chemical and biochemical sensing devices, ther- moelectric, optical, magnetic and electrical application. In this sec- tion, we are focusing on the synthesis of conducting polymer (Section 2.1) and composites (Section 2.2) to develop materials in nanowires architectures.

\subsection{Synthesis of conducting polymer nanowires}



\section{\emph{conclude that NFES is the way to go}}