% Chapter Template

\chapter{Objectives} % Main chapter title

\label{Chapter:Objectives}

%----------------------------------------------------------------------------------------
%	SECTION 1
%----------------------------------------------------------------------------------------



%-----------------------------------
%	SUBSECTION 1
%-----------------------------------


In view of the manufacturing and surface chemistry challenges posed by CNW biosensor
fabrication, the following dissertation is divided in two central objectives:

1) To explore the feasibility of using advanced additive manufacturing techniques to improve the reproducibility of CNW production in the nanoscale range.20 In particular, EMS and Direct Femtosecond Laser Writing (DLW) techniques will be studied and tested in terms of their fabrication parameters. Furthermore, morphological and electrical characterization of the obtained samples will be key in the discussion and assessment of this objective.

2) To study the surface chemistry of CNWs as a material for biomolecule immobilization. The goal will be to explore a functionalization technique that will allow the surface modification of GC to achieve the immobilization of a biomolecule. GC will be studied in particular since it is the predominant material obtained from advanced manufacturing techniques like EMS46 and TPP.50