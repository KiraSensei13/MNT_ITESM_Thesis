% Chapter Template

\chapter{Theoretical Framework} % Main chapter title

\label{Chapter:TheoreticalFramework}

\subsubsection*{\color{mygray}[Chapter under work]}
% Background and state-of-the-art
% This involves a thorough state-ofthe-art bibliographic review, the building of a formal foundation, the study of related techniques, analysis and awareness of previous works and study of fundamental aspects that can help carry on your research.

\section{Electro-Mechanical Spinning}
Diverse polymer patterning techniques have been developed to integrate and synthesize carbon nano-wires. A typical technique is electrospinning. Electrospinning requires the application of an electrostatic force to a polymer solution to spin fibers. The applied electric field, the solution conductivity, jet length, solution viscosity surrounding gas, flow rate and the collector geometry are important parameters that influence the fiber formation during the spinning. \cite{Nataraj2012} Two sub-techniques can be derived from electrospinning depending on the distance between the dispensing electrode and the collector. The process in which the electrospun jet can be controlled near the
tip is called NFES or near-field electrospinning. \cite{Cisquella-Serra2019} Moreover, if the distance between the collector and the dispensing needle is farther, the configuration is known as FFES or far-field electrospinning. \cite{Nataraj2012}

\section{Carbon nano-fibers}
Carbon nano-wires (CMWs) are known as long, thin strings with diameters between 10 and 1 thousand nm; composed mostly by carbon atoms aligned parallel to the long axis of the fiber. \cite{Nataraj2012} Carbon nano-wires are different from carbon tubes, as CMWs are not composed by graphene sheets in cylindrical form. \cite{Nataraj2012}

%----------------------------------------------------------------------------------------
%	SECTION 1
%----------------------------------------------------------------------------------------



%-----------------------------------
%	SUBSECTION 1
%-----------------------------------


