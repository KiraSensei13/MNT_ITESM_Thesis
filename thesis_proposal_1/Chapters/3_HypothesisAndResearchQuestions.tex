% Chapter Template

\chapter{Hypothesis and Research Questions} % Main chapter title

\label{Chapter:HypothesisandResearchQuestions}

%----------------------------------------------------------------------------------------
%	SECTION 1
%----------------------------------------------------------------------------------------



%-----------------------------------
%	SUBSECTION 1
%-----------------------------------


1.2 Research hypotheses

The research hypotheses of the following dissertation are divided into two sections. The first section is related to additive manufacturing techniques used to produce the CNWs; the second one includes the proposed hypotheses with regards to functionalization those structures.

1.2.1 Additive manufacturing

Hypothesis 1 (H1): The geometry (diameter and length) of CNWs produced by ElectroMechanical Spinning (EMS) can be controlled by having consistent fabrication parameters (stage velocity, voltage, dispense rate, polymer formulation). A given set of parameters should, therefore, result in a repeatable nanowire structure with defined geometry. We expect this repeatability will be evident in the statistical deviation from a mean value of the measured diameter and length.

Hypothesis 2 (H2): MPP lithography can be used to produce sub-micrometric CNWs by pyrolysis of a highly cross-linked photoresin, such as SU-8. The dimensions of the produced CNWs are expected to depend on the process variables. By controlling the energy dosage impinged on the photoresin, the localized polymerized volume will vary in size. Therefore, it is expected that dosage will be directly correlated to the diameter of the polymerized polymeric microstructure precursor (and consequently, the diameter of its carbonized version, the CNWs).

1.2.2 Functionalization of carbon nanowires

Hypothesis 3 (H3): Through a suitable surface chemistry modification, CNWs can provide a platform that can serve for biomolecule immobilization.

Hypothesis 4 (H4): Localized surface modification of CNWs can be achieved by chemically protecting the rest of the device and addressing only the wire area.
• Glassy carbon is chemically inert material. We expect this material will be inefficient for biomolecule immobilization.
• By locally depositing a material with stronger affinity to biomolecules, we can study the performance of Suspended CNW as biosensors. In this study, the molecule 4- aminobenzoic acid (4-ABA) will be explored as the interfacing molecule between the surface wire area (composed of GC) and a protein probe.

1.3 Research questions

The following research questions were formulated as a function of hypotheses 1-4:

1. H1: What are the parameters (stage velocity, voltage, dispense rate, polymer formulation) for EMS that allow the fabrication of CNWs? Why have those parameters failed to produce reproducible CNWs in the past? Are there research efforts that suggest the use of an automated deposition of polymeric nanofibers with EMS, which can be pyrolyzed into CNWs?

2. H2: In the literature, is there any evidence of the process parameters that should be employed to fabricate SU-8 polymeric structures with MPP? How does the minimum characteristic features that can be obtained through MPP (i.e. voxels) scale with two and three photon absorption (TPA, 3PA) processes? What are the specific role played by variables such as the intensity of the laser, frequency of operation, writing speed (or stage velocity) in the efficiency of the 2PA/3PA process?

3. H3: What are the most important surface characteristics that allow protein immobilization in an interface? Is there any evidence in the literature that GC can serve as a platform for biomolecule detection? What are the most common functionalization techniques that have been used in carbon nanostructures? What are the most recent techniques and, are they promising?

4. H4: What methods can be employed to achieve local surface modification of CNWs? Are there any reported methods in the literature? Is there a manufacturing method that can allow the selective addressing of the CNW micro/nano-surface, while protecting the rest of the C-MEMS device? Are Electrochemical Modification (ECM) techniques suitable for local modification of GC materials?