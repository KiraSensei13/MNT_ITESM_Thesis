% Chapter Template

\chapter{Problem Definition and Motivation} % Main chapter title

\label{Chapter:ProblemDefinitionandMotivation}

% Describe the current situation of this problem.
% identifying the obstacles and the possible applications of your research.
% background of the problem (e.g. origins, previous attempts, etc.). 
% State the problem in detail mentioning all relative aspects, variables, relationships,
% stressing on the importance of finding a solution to the problem.

Carbon nanowires have been fabricated with a photoresist by multiple-photon polymerization techniques. However little is known about polymers that can produce conductive carbon nanowires after pyrolysis. In the past years photon polymerization processes have been apllied to the fabrication of nano-structures with the use of a photoresist. \cite{Boer2014} Photon polymerization techniques 


This method is a subset of SLA; however, the use of a multiphoton femtosecond laser produces results that are quite unique (Greiner et al., 2012; Ovsianikov et al., 2012). The resolutions are exceptional (Figure 10.15(a)–(c)), with nanoscale tolerances and the ability to produce very small pores and highly detailed structures (Hribar et al., 2014). Similar to that of SLA, a photocurable resin is required and currently only small objects can be produced and the resins are often toxic (Figure 10.15(d)–(f)). However, these limitations are well known and there are groups working intensively to overcome these challenges. Ovsianikov et al. (2012) demonstrated 3D photografting of a PEG-based matrix to produce functionalized regions within the hydrogel. Laza et al. (2012) also performed two-photon polymerization within lamellar fluid flow (Figure 10.15(g)–(i)). Without turbulence, the precise flow allows a small structure such as a spring to be produced in a continuous process. While not built as an entire scaffolding structure, fabricating elements that later can be combined into a scaffold is an exciting prospect.



%----------------------------------------------------------------------------------------
%	SECTION 1
%----------------------------------------------------------------------------------------



%-----------------------------------
%	SUBSECTION 1
%-----------------------------------

