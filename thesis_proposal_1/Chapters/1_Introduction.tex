% Chapter Template

\chapter{Introduction} % Main chapter title

\label{Chapter:Introduction}

% Introduction to the Context where the proposal is going to be carried out.
Carbon nano-materials are subjected to great interest for research purposes due to their various potential applications in diverse areas that take advantage of the nano-scale properties. Carbon nano-materials are suitable for the catalysis, adsorption, carbon capture, energy and hydrogen storage, drug delivery, bio-sensing and cancer detection. \cite{Siddiqui2019} Due to some of matchless properties that allow carbon nano-materials to be utilized within multiple functionalities including high porosity, distinguished structures, uniform morphologies, high stability, high magnetic properties and high conductivity. \cite{Siddiqui2019}

% Describe the Problematic Situation
% Briefly describe the problem, the relevant aspects and factors that are part of it.
% Justifify the relevance to provide a solution to the problem.
% Explain what has been previoulsy done to find a solution to the problem.
% Describe the proposed solution model
% State the expected achievements by solving this problema.
% Describe the organization of the document.

This document bestow a thesis proposal to perform a research to engineer and design a polymer solution to achieve mass scale manufacturing of carbon nano-wires in a cheap, continuous, simple and reproducible manner. The research intends to involve several manufacturing processes such as near field electrospinning, photopolymerization, pyrolization and carbonization, as they have shown to be promising methods for synthesis of carbon nano-materials. \cite{Cardenas2017} A number of processes have been developed for specific purposes of polymeric nano-fibers, some include surface deposition, composites, and chemical adjustments. Polymeric nano-fibers could be also pyrolyzed to generate carbon nano-wires with conductive capabilities \cite{Madou2011} for electrochemical sensing and energy storage purposes.

Nanotechnology has explored different polymer patterning techniques to integrate carbon nano-wires structures. One technique is known as far-field electrospinning, a process in which electrified jets of polymer solution are jettisoned to synthesize nano-fibers which then are pyrolyzed at high temperatures. One sub-technique derived from electrospinning is near field electromechanical spinning or EMS. EMS has proved to deliver high control in patterning polymeric nano-fibers. \cite{Cardenas2017}

The proposal is to continue the previous work done in regards of the synthesis of carbon nano-wires. Previous work includes the fabrication of suspended carbon nano-wires by two methods: electro-mechanical spinning and multiple-photon polymerization with a photoresist. \cite{Cardenas2017} This proposal research is intended to focus on electro-mechanical spinning processes only, to bring off polymer solutions that can be electrospun by nier field electrospinning (NFES), photopolymerized and pyrolyzed into conducting carbon nano-wires. The polymer solutions described in \cite{Cardenas2017} are to be amended to achieve the goal mentioned in the previous statement.

Near-field electrospinning or NFES allows large scale manufacturability combined with controlled guidance. \cite{Madou2011} However, the reported efforts required the use of electric fields in excess of 200 kV/m for continuous operation so that the resulting in limited control for nano-fiber patterning. \cite{Madou2011} the current state-of-the-art synthesis processes for polymer nano-fibers lack to yield precise, inexpensive, fast, and continuous manufacturing properties.

%----------------------------------------------------------------------------------------
%	SECTION 1
%----------------------------------------------------------------------------------------



%-----------------------------------
%	SUBSECTION 1
%-----------------------------------


