%%
%% Copyright 2007, 2008, 2009 Elsevier Ltd
%%
%% This file is part of the 'Elsarticle Bundle'.
%% ---------------------------------------------
%%
%% It may be distributed under the conditions of the LaTeX Project Public
%% License, either version 1.2 of this license or (at your option) any
%% later version.  The latest version of this license is in
%%    http://www.latex-project.org/lppl.txt
%% and version 1.2 or later is part of all distributions of LaTeX
%% version 1999/12/01 or later.
%%
%% The list of all files belonging to the 'Elsarticle Bundle' is
%% given in the file `manifest.txt'.
%%
\documentclass[5p,,preprint,12pt,twocolumn]{elsarticle}
\makeatletter\if@twocolumn\PassOptionsToPackage{switch}{lineno}\else\fi\makeatother


\usepackage{tabulary,xcolor}
\usepackage{amsfonts,amsmath,amssymb}
\usepackage[T1]{fontenc}
\makeatletter
\let\save@ps@pprintTitle\ps@pprintTitle
\def\ps@pprintTitle{\save@ps@pprintTitle\gdef\@oddfoot{\footnotesize\itshape \null\hfill\today}}
\def\hlinewd#1{%
  \noalign{\ifnum0=`}\fi\hrule \@height #1%
  \futurelet\reserved@a\@xhline}
\def\tbltoprule{\hlinewd{.8pt}\\[-12pt]}
\def\tblbottomrule{\noalign{\vspace*{6pt}}\hline\noalign{\vspace*{2pt}}}
\def\tblmidrule{\noalign{\vspace*{6pt}}\hline\noalign{\vspace*{2pt}}}
\AtBeginDocument{\ifNAT@numbers \biboptions{sort&compress}\fi}
\makeatother

  


\usepackage{ifluatex}
\ifluatex
\usepackage{fontspec}
\defaultfontfeatures{Ligatures=TeX}
\usepackage[]{unicode-math}
\unimathsetup{math-style=TeX}
\else 
\usepackage[utf8]{inputenc}
\fi 
\ifluatex\else\usepackage{stmaryrd}\fi

  
%%%%%%%%%%%%%%%%%%%%%%%%%%%%%%%%%%%%%%%%%%%%%%%%%%%%%%%%%%%%%%%%%%%%%%%%%%
% Following additional macros are required to function some 
% functions which are not available in the class used.
%%%%%%%%%%%%%%%%%%%%%%%%%%%%%%%%%%%%%%%%%%%%%%%%%%%%%%%%%%%%%%%%%%%%%%%%%%
\usepackage{url,multirow,morefloats,floatflt,cancel,tfrupee}
\makeatletter


\AtBeginDocument{\@ifpackageloaded{textcomp}{}{\usepackage{textcomp}}}
\makeatother
\usepackage{colortbl}
\usepackage{xcolor}
\usepackage{pifont}
\usepackage[nointegrals]{wasysym}
\urlstyle{rm}
\makeatletter

%%%For Table column width calculation.
\def\mcWidth#1{\csname TY@F#1\endcsname+\tabcolsep}

%%Hacking center and right align for table
\def\cAlignHack{\rightskip\@flushglue\leftskip\@flushglue\parindent\z@\parfillskip\z@skip}
\def\rAlignHack{\rightskip\z@skip\leftskip\@flushglue \parindent\z@\parfillskip\z@skip}

%Etal definition in references
\@ifundefined{etal}{\def\etal{\textit{et~al}}}{}


%\if@twocolumn\usepackage{dblfloatfix}\fi
\usepackage{ifxetex}
\ifxetex\else\if@twocolumn\@ifpackageloaded{stfloats}{}{\usepackage{dblfloatfix}}\fi\fi

\AtBeginDocument{
\expandafter\ifx\csname eqalign\endcsname\relax
\def\eqalign#1{\null\vcenter{\def\\{\cr}\openup\jot\m@th
  \ialign{\strut$\displaystyle{##}$\hfil&$\displaystyle{{}##}$\hfil
      \crcr#1\crcr}}\,}
\fi
}

%For fixing hardfail when unicode letters appear inside table with endfloat
\AtBeginDocument{%
  \@ifpackageloaded{endfloat}%
   {\renewcommand\efloat@iwrite[1]{\immediate\expandafter\protected@write\csname efloat@post#1\endcsname{}}}{\newif\ifefloat@tables}%
}%

\def\BreakURLText#1{\@tfor\brk@tempa:=#1\do{\brk@tempa\hskip0pt}}
\let\lt=<
\let\gt=>
\def\processVert{\ifmmode|\else\textbar\fi}
\let\processvert\processVert

\@ifundefined{subparagraph}{
\def\subparagraph{\@startsection{paragraph}{5}{2\parindent}{0ex plus 0.1ex minus 0.1ex}%
{0ex}{\normalfont\small\itshape}}%
}{}

% These are now gobbled, so won't appear in the PDF.
\newcommand\role[1]{\unskip}
\newcommand\aucollab[1]{\unskip}
  
\@ifundefined{tsGraphicsScaleX}{\gdef\tsGraphicsScaleX{1}}{}
\@ifundefined{tsGraphicsScaleY}{\gdef\tsGraphicsScaleY{.9}}{}
% To automatically resize figures to fit inside the text area
\def\checkGraphicsWidth{\ifdim\Gin@nat@width>\linewidth
	\tsGraphicsScaleX\linewidth\else\Gin@nat@width\fi}

\def\checkGraphicsHeight{\ifdim\Gin@nat@height>.9\textheight
	\tsGraphicsScaleY\textheight\else\Gin@nat@height\fi}

\def\fixFloatSize#1{}%\@ifundefined{processdelayedfloats}{\setbox0=\hbox{\includegraphics{#1}}\ifnum\wd0<\columnwidth\relax\renewenvironment{figure*}{\begin{figure}}{\end{figure}}\fi}{}}
\let\ts@includegraphics\includegraphics

\def\inlinegraphic[#1]#2{{\edef\@tempa{#1}\edef\baseline@shift{\ifx\@tempa\@empty0\else#1\fi}\edef\tempZ{\the\numexpr(\numexpr(\baseline@shift*\f@size/100))}\protect\raisebox{\tempZ pt}{\ts@includegraphics{#2}}}}

%\renewcommand{\includegraphics}[1]{\ts@includegraphics[width=\checkGraphicsWidth]{#1}}
\AtBeginDocument{\def\includegraphics{\@ifnextchar[{\ts@includegraphics}{\ts@includegraphics[width=\checkGraphicsWidth,height=\checkGraphicsHeight,keepaspectratio]}}}

\DeclareMathAlphabet{\mathpzc}{OT1}{pzc}{m}{it}

\def\URL#1#2{\@ifundefined{href}{#2}{\href{#1}{#2}}}

%%For url break
\def\UrlOrds{\do\*\do\-\do\~\do\'\do\"\do\-}%
\g@addto@macro{\UrlBreaks}{\UrlOrds}



\edef\fntEncoding{\f@encoding}
\def\EUoneEnc{EU1}
\makeatother
\def\floatpagefraction{0.8} 
\def\dblfloatpagefraction{0.8}
\def\style#1#2{#2}
\def\xxxguillemotleft{\fontencoding{T1}\selectfont\guillemotleft}
\def\xxxguillemotright{\fontencoding{T1}\selectfont\guillemotright}

\newif\ifmultipleabstract\multipleabstractfalse%
\newenvironment{typesetAbstractGroup}{}{}%

%%%%%%%%%%%%%%%%%%%%%%%%%%%%%%%%%%%%%%%%%%%%%%%%%%%%%%%%%%%%%%%%%%%%%%%%%%
\emergencystretch 20pt \tolerance = 1500 \def\floatpagefraction{0.8}




%%%%%%%%%%%%%%%%%%%%%%%%%%%%%%%%%%%%%%%%%%
% Feature enabled:
%pagenum: yes
%text-layout: twocolumn
%%%%%%%%%%%%%%%%%%%%%%%%%%%%%%%%%%%%%%%%%%

\makeatletter
\def\ps@pprintTitle{\save@ps@pprintTitle\gdef\@oddfoot{\footnotesize\hspace*{.5\textwidth}\thepage\itshape \null\hfill\today}}
\makeatother
          
\usepackage{longtable}

\usepackage{float}

\makeatletter
\AtBeginDocument{\@ifpackageloaded{rotating}{\PassOptionsToPackage{figuresright}{rotating}}{\usepackage[figuresright]{rotating}}\setlength{\rotFPtop}{0pt plus 1fil}\setlength{\rotFPbot}{0pt plus 1fil}}
\makeatother

\makeatletter
 \AtBeginDocument{%
  \@ifpackagewith{endfloat}{figuresonly}
  {\DeclareDelayedFloatFlavor{sidewaysfigure}{figure}}%true
  {\@ifpackagewith{endfloat}{tablesonly}{\DeclareDelayedFloatFlavor{sidewaystable}{table}\DeclareDelayedFloatFlavor{longtable}{table}\DeclareDelayedFloatFlavor{landscape}{table}}%true
  {\@ifpackageloaded{endfloat}{\DeclareDelayedFloatFlavor{sidewaysfigure}{figure}\DeclareDelayedFloatFlavor{sidewaystable}{table}\DeclareDelayedFloatFlavor{longtable}{table}\DeclareDelayedFloatFlavor{landscape}{table}}{}}%false
  }%false
  }
\makeatother

\usepackage{pdflscape}

\begin{document}



\begin{frontmatter}
	
\title{Review of Polymer Solutions for Near-Field Electrospinning with Spatial Control
}
    
\author[]{Antonio Osamu Katagiri Tanaka}
\ead{oskatagiri@gmail.com}
\author[]{H{\'e}ctor Al\'{a}n Aguirre Soto}
\ead{alan.aguirre@tec.mx}
    

\begin{abstract}
Near-field electrospinning (NFES) is identified to be a technique able to fabricate polymer nano and micro fibers with accurate placement. In the past years (2006-2019), several polymer solutions have been successfully electrospun into fibers through several variants of the conventional NFES process. Each NFES variant intents to tailor the process parameters in order to improve the fibers' properties. This paper presents a review on the research and related development of electrospun fibers, emphasizing the used polymers, solvents, and fiber characteristics. Relevant summary of polymer solutions and near-field electrospinning processing conditions is provided in this paper.
\end{abstract}
\begin{keyword} 
      polymer\sep solvent\sep near-field electrospinning\sep NFES\sep fibers\sep spatial control
\end{keyword}
      
\end{frontmatter}
    
\section{Introduction}
Even though electrospinning is an old invention \unskip~\cite{527120:12073288}, it is currently a trending topic among researchers \unskip~\cite{527120:12073453,527120:12073495,527120:12073496}. One of the reasons electrospinning is to be studied is its potential to fabricate polymer nano-fibers from a variety of polymers. The technique allows the production of thin continuous fibers with ease, with diameters down to 3 $nm $ in some cases, which is something difficult to achieve by other techniques. Furthermore, the basic setup can be modified with ease to fabricate different fibers with diversified functionalities with different materials. The produced fibers can be aligned or unaligned. Besides, the electrospinning equipment is inexpensive and of small size, compared to the equipment of standard spinning techniques. On the other hand, the understanding of the electrospinning process has improved in the last years \unskip~\cite{527120:12073538}.

The main components of the electrospinning technique are the fluid control unit (e.g. syringe pump) and a voltage power supply. The process also requires a target electrode or combination of electrodes on which the fibers can be collected.  Figure~\ref{f-fe28447572e9} describes a typical near-field electrospinning set-up \unskip~\cite{527120:12073538}. Two sub-techniques can be derived from electrospinning depending on the distance between the dispensing electrode and the collector. The process in which the electrospun jet can be controlled near the tip is called NFES or near-field electrospinning \unskip~\cite{527120:12033655}. Moreover, if the distance between the collector and the dispensing needle is greater, the configuration is known as FFES or far-field electrospinning \unskip~\cite{527120:12073581}.


\bgroup
\fixFloatSize{images/596d6818-a246-493e-bea0-03f815ab8ff8-unfes.jpg}
\begin{figure*}[!htbp]
\centering \makeatletter\IfFileExists{images/596d6818-a246-493e-bea0-03f815ab8ff8-unfes.jpg}{\includegraphics{images/596d6818-a246-493e-bea0-03f815ab8ff8-unfes.jpg}}{}
\makeatother 
\caption{{Typical near-field electrospinning set-up \unskip~\protect\cite{527120:11973130} .}}
\label{f-fe28447572e9}
\end{figure*}
\egroup
Near-field electrospinning is considered to be an outstanding technique to fabricate polymer fibers with spatial control and it has suffered several modifications to improve the precision and accuracy of the fiber deposition. This paper intents to collect the NFES variants of electrospunable polymer solutions with spatial control in recent research.
    
\section{Polymer Solution}
In electrospinning, it is generally agreed that with higher concentration, the diameter of the fibers increased due to greater viscosity which resist stretching. In near field electrospinning, similar observations have been reported where concentration increases, fiber diameter increased\unskip~\cite{527120:11974306,527120:11974329}. However, in separate studies by Pan et al.\unskip~\cite{527120:11974317,527120:12321129} using poly(\ensuremath{\gamma }-benzyl \ensuremath{\alpha }, l-glutamate) and polyvinylidene fluoride (PVDF) reported reduction in fiber diameter with increasing concentration. Pan et al.\unskip~\cite{527120:12321129} attributed this to a higher charge accumulation in higher concentration PVDF solution. However, more studies need to be carried out to verify this.



\subsection{Polymers}[SECTION UNDERWORK]



\subsection{Solvents}[SECTION UNDERWORK]
    
\section{NFES Parameters}
To spin nano fibers at close distances, the initial diameter of the jet is required to be as small as possible since stretching of the thread is limited. Kameoka et al.\unskip~\cite{527120:12321556} demonstrated that a small initial spinning radius can be achieved using an atomic force microscope tip with a small polymer solution drop at the tip.

Near-field electrospinning, has exhibited to be capable fabricate nano fibers over and nano fiber patterns \unskip~\cite{527120:11974321}. Nevertheless, having a small polymer solution drop at the nozzle tip limits the length of the fibers that can be fabricated in a continuous manner. Using a spinneret with a reservoir (e.g. syringe) of solution generally produces fibers with diameter of a few micrometers \unskip~\cite{527120:11974310,527120:11974326}, since it creates a limit to which the nozzle inner diameter can be reduced to allow the solution to flow through.

Coppola et al.\unskip~\cite{527120:11974307} have showed a NFES variant that allows polymer nano fibers to be deposited directly from a polymer drop, averting the issue of nozzle clogging. The fibers are also prone soaking after deposition thus giving the fibers a semi-circular cross-section as depicted in Xue et al.'s\unskip~\cite{527120:11974326} work. 

The thinnest nozzles in literature so far are about 100 $\mu m $ in diameter, for instance Chang et al.\unskip~\cite{527120:11974306} used a 100 $\mu m $ inner diameter needle tip to electrospin poly(ethylene oxide) (PEO) and Camillo et al.\unskip~\cite{527120:12322072} used a micro-diameter tip Tungsten spinneret in a 26G needle to electrospin co-polymer, poly[2-methoxy-5-(2-ethylhexyloxy)-1,4-phenylenevinylene] (MEH-PPV) with poly(ethylene oxide) (PEO).



\subsection{Applied Voltage}In recent literature, near field electrospinning has been studied to reduce the fiber diameter and to improve the fiber deposition accuracy. Camillo et al.\unskip~\cite{527120:12322072} demonstrated that the application of a modified fine tip nozzle enables the fabrication of 100 $nm $ diameter fiber at a nozzle-to-substrate distance of 500 $\mu m $ and an applied voltage of 1.5 $kV $ . On the other hand, Bisht et al.\unskip~\cite{527120:11973130} and Chang et al.\unskip~\cite{527120:11974306} came to the conclusion higher voltages yield thicker micro-fibers with a loss in jet stability.

This discrepancy in literature between the applied voltage and resulting fiber diameter is due to the relationship with other variables such as nozzle-to-substrate distance and solution deposition rate. For instance, if a high voltage is applied at a low deposition rate then electrospraying is achieved, meaning the formation of several non-continuous fibers. The applied voltage shall be sufficient to break the surface tension and initiate the jet, but low enough to avoid multiple jets at the nozzle tip.

Bisht et al.\unskip~\cite{527120:11973130} achieved the fabrication of thinner fibers with spatial control by reducing the applied voltage to 200-600 $V $  at a nozzle-to-substrate distance of 0.5-1 $mm $. The low voltage setting does not create enough charge to break the polymer solution surface tension to initiate the electrospinning process.

Bisht et al.\unskip~\cite{527120:11973130} and Chang et al.\unskip~\cite{527120:11974306} initiated the electrospun fibers by mechanically pull the polymer solution at the nozzle tip using a micro-probe tip. Chang and coworkers reduced the applied voltage from 1.5 $kV $ to 600 $V $ with a nozzle-to-substrate distance of 500 $\mu m $ to yield a fiber diameter between 3 $\mu m $  and 50 $nm $ . With an applied voltage of 200 $V $ and a nozzle-to-substrate distance of 1 $mm $ , PEO nano fibers were deposited with a diameter about 20 $nm $.

In near-field electrospinning, the applied voltage has an impact on the produced fiber morphology. For instance, a voltage higher or lower to the optimum voltage will translate into an increase in fiber diameter. Song et al.\unskip~\cite{527120:11974320} demonstrated that a decrease in voltage from 400 to 500 $V $ can reduce the fiber diameter from 160 to about 60 $nm $with a nozzle-to-substrate distance of 20 $\mu m $. The optimum voltage is achieved when a balance is attained between the stretching of the jet and the speed at which it hits the substrate. The increase of voltage yields thinner fibers as it causes greater stretching, and a greater jet acceleration.

Another workaround to break the polymer solution surface tension is to initialize the NFES process with a higher voltage and then lower the voltage once the jet is created. Huang et al.\unskip~\cite{527120:11974311} implemented the previous and yield ordered fibers with a distance between adjacent fibers of 50 $\mu m $. In most cases, a positive voltage is applied to the spinneret.



\subsubsection{Nozzle-to-substrate distance}In NFES, the fiber morphology can be altered by the control of the height between the nozzle and the substrate (collector). With the decrease of the nozzle-to-substrate distance, the electric field strength increases; however it can cause incomplete solvent volatilisation and possible short circuits between the collector and the nozzle tip.

An optimal nozzle-to-substrate distance shall be defined to ensure the fabrication of dry continuous fibers. If the solvent is not well evaporated, the produced fibers are prone to defects; on the other hand if solidification happens too fast, the solids can block the spinneret which can prevent a continuous fiber yield. Furthermore, the polymer jet will discharge itself as soon as possible, therefore long distances can result in low yields.

Typically, metal nozzle tips are used, with small inner diameters. From literature, needles with small diameters produce thinner fibers. A thin nozzle tip can help the reduction of the fiber diameter, but also it is more likely to become blocked.



\subsubsection{Electric field}Recent literature suggests that the fiber morphology depends on the electric field profile created by the applied voltage during NFES. Since the electric field is an induced force that attracts the solution jet towards the desired location within the collector.

Bisht et al.\unskip~\cite{527120:11973130} and Min et al.\unskip~\cite{527120:11974316} have reported the ability to electrospin nano fibers with high accuracy. Min et al. \unskip~\cite{527120:11974316} implemented a NFES setup with multiple "field-effect transistors" on a flexible polyacrylate collector with an x-y stage velocity of 13.3 $cm/s $ to fabricate fibers with a diameter about 289 $nm $ and a distance between adjacent fibers of 50 $\mu m $. 

On the other hand, Bisht et al.\unskip~\cite{527120:11973130} showed evidence of fabricated fibers with low-voltage NFES with high accuracy and precision. Bisht et al.'s suspended fibers were deposited over carbon posts with a distance between adjacent fibers of 100 $\mu m $ with diameter of 30 $\mu m $\unskip~\cite{527120:11973130}.

The employment of guided electrodes in NFES, adapts the fabrication process to yield a more accurate fiber deposition. For instance, Kim et al.\unskip~\cite{527120:11974313} manufactured ink patterns on a paper with silver nano particles. The printed patterns aid the fibers to land on the desired location. Kim et al.\unskip~\cite{527120:11974313} electrospun the fibers with a distance between adjacent fibers of 150 $\mu m $.

Xu et al.\unskip~\cite{527120:11974325} created a straight jet from the nozzle tip to the substrate using a guiding electrode underneath the collector. The purpose of the guiding electrode is to adjust the path of the NFES jet. With the guiding electrode implementation, the fiber's spread was reduced from 74 $\mu m $ to 7 $\mu m $.



\subsection{Substrate}Due to the close distance between the grounded substrate and the charged spinneret in NFES, the set up is prone to electrical shorts. In NFES, when a short circuit takes place, the electrospinning process is interrupted resulting in the fabrication of discontinuous fibers. Two workarounds to avoid electrical shorts is to lower the applied voltage and to install less conductive substrates \unskip~\cite{527120:11974315,527120:12322289}.

Liu et al.\unskip~\cite{527120:11974315} discovered that the fiber alignment is improved by using a glass-cooper foil substrate, however the well aligned fibers are spoiled after prolonged depositions due to residual charges. Additionally, the effect of residual charges is amplified with the used collector substrate contains a conductive layer and a non-conductive layer\unskip~\cite{527120:11974315}.

On the other hand, Choi et al.\unskip~\cite{527120:12322289} implemented a hydrophobic substrate to deposit the fibers with plasma treatment to increase the conductivity of selected areas. NFES was carried put with precise deposition as the fibers were placed as per the desired design within the hydrophilic substrate.
\begin{landscape}
\makeatletter\@twocolumnfalse\makeatother
\begingroup
\makeatletter\if@twocolumn\@ifundefined{theposttbl}{\gdef\TwoColDocument{true}\onecolumn\onecolumn}{}\fi\makeatother \setlength\LTcapwidth{\textheight}
\begin{longtable}{p{\dimexpr.1641\linewidth-2\tabcolsep}p{\dimexpr.13520000000000005\linewidth-2\tabcolsep}p{\dimexpr.1662\linewidth-2\tabcolsep}p{\dimexpr.46839999999999996\linewidth-2\tabcolsep}p{\dimexpr.0661\linewidth-2\tabcolsep}}
\caption{{Electrospun Polymer Solutions - Solution and Process Parameters} }
\label{tw-bab4042dace7}
\def\arraystretch{1}\\\endfirsthead \hline \noalign{\vskip3pt} \noalign{\textit{Table \thetable\ continued}} \noalign{\vskip3pt} \hline \endhead \hline \noalign{\vskip3pt} \noalign{\textit{\hfill Continued on next page}} \noalign{\vskip3pt} \endfoot \endlastfoot 
\tbltoprule Polymer(s) & Solvent(s) & NFES Variant & Process Parameters and Fiber Characterization & Ref.\\
\tblmidrule 
Poly(ethylene oxide) (PEO; MW = 4,000,000) &
  Deionized water &
  Low-Voltage NFES (LV NFES) &
  \textbf{Solution Concentration:} 1, 2, and 3 $wt\% $ PEO \mbox{}\protect\newline \textbf{Nozzle:} 27 gauge type 304; stainless steel needle \mbox{}\protect\newline \textbf{Solution deposition rate:} lower than 1$\mu L / h $ \mbox{}\protect\newline \textbf{Nozzle-to-substrate distance:} 1$mm $ \mbox{}\protect\newline \textbf{Substrate composition: }Pyrolyzed SU-8 carbon and Si \mbox{}\protect\newline \textbf{Applied voltage: }polymer jet initiated at 400-600 $V $ and dispensed at 200-400 $V $ \mbox{}\protect\newline \textbf{x-y stage velocity:} 10-40$mm/s $ \mbox{}\protect\newline \textbf{Fiber Diameter:} 50-425$nm $ \mbox{}\protect\newline \textbf{Distance between adjacent fibers:} \textit{Not determined} &
  \unskip~\cite{527120:11973130}\\\cline{1-1}\cline{2-2}\cline{3-3}\cline{4-4}\cline{5-5}
Poly[2-methoxy-5-(2-ethylhexyloxy)-1,4-phenylenevinylene] (MEH-PPV; MW = 380,000) with Poly(ethylene oxide) (PEO; MW = 300,000) &
  acetonitrile toluene mixture (65/35); acetic acid toluene (17/83); pure toluene &
  Typical NFES process &
  \textbf{Solution Concentration:} \mbox{}\protect\newline 10$mg $ of MEH-PPV in 2$mL $ of toluene; 500$mL $ of MEH-PPV solution with 250$mg $ of PEO in 3.5$mL $ of acetonitrile; 500$mL $ of MEH-PPV solution with 250$mg $ of PEO in 3$mL $ of acetic acid / toluene (17 / 83). The resulting MEH-PPV/PEO concentration is 1:100 \mbox{}\protect\newline \textbf{Nozzle:} mm-diameter tip Tungsten spinneret in a 26 gauge needle \mbox{}\protect\newline \textbf{Solution deposition rate:} 50$\mu L / h $ \mbox{}\protect\newline \textbf{Nozzle-to-substrate distance:} 500$\mu m $ \mbox{}\protect\newline \textbf{Substrate composition:} SiO2/Si (oxide thickness = 800 nm) \mbox{}\protect\newline \textbf{Applied voltage:} around 1.3$kV $ \mbox{}\protect\newline \textbf{x-y stage velocity:} 50$cm/s $ \mbox{}\protect\newline \textbf{Fiber Diameter:} 100$nm $ \mbox{}\protect\newline \textbf{Distance between adjacent fibers:} around 100$\mu m $ &
  \unskip~\cite{527120:11974305}\\\cline{1-1}\cline{2-2}\cline{3-3}\cline{4-4}\cline{5-5}
Poly(ethylene oxide) (PEO) &
  Water &
  Scanning Tip Electrospinning and NFES &
  \textbf{Solution Concentration:} 7$wt\% $ PEO \mbox{}\protect\newline \textbf{Nozzle:} Needle outer diameter of 200$\mu m $ and inner diameter of 100$\mu m $ \mbox{}\protect\newline \textbf{Solution deposition rate:} 0.1$\mu L / h $ \mbox{}\protect\newline \textbf{Nozzle-to-substrate distance:} 500$\mu m $ \mbox{}\protect\newline \textbf{Substrate composition:} \textit{Not determined} \mbox{}\protect\newline \textbf{Applied voltage:} polymer jet initiated at 1.5 $kV $ and dispensed at 600$V $ \mbox{}\protect\newline \textbf{x-y stage velocity:} 120$mm/s $ \mbox{}\protect\newline \textbf{Fiber Diameter:} 709$\pm $131$nm $; 49-74$nm $ when applied voltage is 800$V $ \mbox{}\protect\newline \textbf{Distance between adjacent fibers:} \textit{ Not determined} \mbox{}\protect\newline \textbf{Notes:} 108$m $ yield in 15$min $ with a fiber diameter of 709$\pm $131$nm $ &
  \unskip~\cite{527120:11974306}\\\cline{1-1}\cline{2-2}\cline{3-3}\cline{4-4}\cline{5-5}
Poly(vinylidine fluorid) (PVDF) &
  N,N Dimethylformamide (DMF) &
  Helix Electrohydro-dynamic Printing (HE-printing) &
  \textbf{Solution Concentration:} 1.8$g $ PVDF in 4.1$g $ of DMF and 4.1$g $ of acetone. The resulting concentration is 18\% PVDF. \mbox{}\protect\newline \textbf{Nozzle:} Needle outer diameter of 510$\mu m $ and inner diameter of 260$\mu m $ \mbox{}\protect\newline \textbf{Solution deposition rate:} 400$nL/min $ \mbox{}\protect\newline \textbf{Nozzle-to-substrate distance:} 10-50$mm $ \mbox{}\protect\newline \textbf{Substrate composition: }Poly(dimethylsiloxane) (PDMS) on Ecoflex \mbox{}\protect\newline \textbf{Applied voltage:} 1.5{\textendash}3$kV $ \mbox{}\protect\newline \textbf{x-y stage velocity:} 0-400$mm/min $ \mbox{}\protect\newline \textbf{Fiber Diameter:} about 1.5-3$\mu m $ \mbox{}\protect\newline \textbf{Distance between adjacent fibers:} \textit{Not determined} &
  \unskip~\cite{527120:11974308}\\\cline{1-1}\cline{2-2}\cline{3-3}\cline{4-4}\cline{5-5}
Polyhedral Oligomeric Silsesquioxane-Poly(Carbonate-Urea)Urethane (POSS-PCU) and Polyhedral Oligomeric Silsesquioxane Poly(Caprolactone-Poly(Carbonate-Urea)Urethane) (POSS-PCL-PCU) &
  Dimethyl acetamide (DMAC) and 1-Butanol &
  Electrohydro-dynamic 3D Print-patterning or Electrohydro-dynamic Jetting &
  \textbf{Solution Concentration: }POSS-PCU and POSS-PCL-PCU used in 20\%$w/w $ concentration in DMAC \mbox{}\protect\newline \textbf{Nozzle:} needle of 750 $\mu m $ in diameter \mbox{}\protect\newline \textbf{Solution deposition rate:} less than 1$\mu L / min $ \mbox{}\protect\newline \textbf{Nozzle-to-substrate distance: }about between 500$\mu m $ to 2$mm $ \mbox{}\protect\newline \textbf{Substrate composition:} \textit{Not determined} \mbox{}\protect\newline \textbf{Applied voltage:} 8.0-10.0$kV $ \mbox{}\protect\newline \textbf{x-y stage velocity:} 10$mm/s $ \mbox{}\protect\newline \textbf{Fiber Diameter:} 5-50$\mu m $ \mbox{}\protect\newline \textbf{Distance between adjacent fibers: }250$\mu m $ &
  \unskip~\cite{527120:11974310}\\\cline{1-1}\cline{2-2}\cline{3-3}\cline{4-4}\cline{5-5}
Poly(ethylene oxide) (PEO) &
  Distilled water &
  Electrohydro-dynamic Writing or Mechanoelectrospinning (MES) &
  \textbf{Solution Concentration:} 6$wt\% $ PEO \mbox{}\protect\newline \textbf{Nozzle:} \textit{Not determined} \mbox{}\protect\newline \textbf{Solution deposition rate:} 1200$nL/min $ \mbox{}\protect\newline \textbf{Nozzle-to-substrate distance:} 7.5$mm $ \mbox{}\protect\newline \textbf{Substrate composition:} \textit{Not determined} \mbox{}\protect\newline \textbf{Applied voltage:} polymer jet initiated at 2 $kV $ and dispensed at 0.8-1$kV $ \mbox{}\protect\newline \textbf{x-y stage velocity:} around 400$mm/s $ \mbox{}\protect\newline \textbf{Fiber Diameter:} 200-350$nm $ \mbox{}\protect\newline \textbf{Distance between adjacent fibers:} 5$\mu m $ &
  \unskip~\cite{527120:11974311}\\\cline{1-1}\cline{2-2}\cline{3-3}\cline{4-4}\cline{5-5}
Poly(ethylene oxide) (PEO) &
  Deionized water and the ethanol with a volume ratio of 3:1 &
  Airflow-assisted Electrohydro-dynamic Direct-writing (EDW) &
  \textbf{Solution Concentration:} 8$wt\% $ PEO \mbox{}\protect\newline \textbf{Nozzle:} Outer airflow passage diameter: 1$mm $ Airflow gas pump pressure: 25$kPa $ Inner liquid passage diameter: 0.21$mm $ \mbox{}\protect\newline \textbf{Solution deposition rate:} 30$\mu L / h $ \mbox{}\protect\newline \textbf{Nozzle-to-substrate distance:} 2$mm $ \mbox{}\protect\newline \textbf{Substrate composition: }Silicon \mbox{}\protect\newline \textbf{Applied voltage:} about 2$kV $ \mbox{}\protect\newline \textbf{x-y stage velocity:} 1-20$mm/s $ \mbox{}\protect\newline \textbf{Fiber Diameter:} 3.73 $\pm $ 1.37$\mu m $ \mbox{}\protect\newline \textbf{Distance between adjacent fibers: }5.13 $\pm $ 6.67$\mu m $ &
  \unskip~\cite{527120:11974312}\\\cline{1-1}\cline{2-2}\cline{3-3}\cline{4-4}\cline{5-5}
Poly(Vinylidene Fluoride) (PVDF) &
  Acetone and Dimethyl Sulfoxide (DMSO) &
  3D Electrospinning &
  \textbf{Solution Concentration:} 17$wt\% $ PVDF; 1.7$g $ of PVDF, 5$g $ of acetone, 0.5$g $ of Capstone FS-66, 5$g $ of DMSO \mbox{}\protect\newline \textbf{Nozzle:} Needle inner diameter of 100$\mu m $ \mbox{}\protect\newline \textbf{Solution deposition rate:} 14$\;nL/min $ \mbox{}\protect\newline \textbf{Nozzle-to-substrate distance:} 750$\mu m $ \mbox{}\protect\newline \textbf{Substrate composition:} A4 size commercial printing paper (Double A) \mbox{}\protect\newline \textbf{Applied voltage:} 1.9$kV $ \mbox{}\protect\newline \textbf{x-y stage velocity:} 10$mm/s $ \mbox{}\protect\newline \textbf{Fiber Diameter:} \textit{Not determined} \mbox{}\protect\newline \textbf{Distance between adjacent fibers:} \textit{Not determined} &
  \unskip~\cite{527120:11974313}\\\cline{1-1}\cline{2-2}\cline{3-3}\cline{4-4}\cline{5-5}
Poly(9-Vinyl Carbazole) (PVK) &
  Styrene &
  Typical NFES process &
  \textbf{Solution Concentration:} 3.96$wt\% $ PVK in styrene \mbox{}\protect\newline \textbf{Nozzle:} Needle inner diameter of 100$\mu m $ \mbox{}\protect\newline \textbf{Solution deposition rate:} 500$nL/min $ \mbox{}\protect\newline \textbf{Nozzle-to-substrate distance:} around 2.5$mm $ \mbox{}\protect\newline \textbf{Substrate composition:} Si/SiO2 \mbox{}\protect\newline \textbf{Applied voltage:} 3-4$kV $ \mbox{}\protect\newline x-y stage velocity: 13.3$cm/s $ \mbox{}\protect\newline \textbf{Fiber Diameter:} 289.26 $\pm $ 35.37$nm $ \mbox{}\protect\newline \textbf{Distance between adjacent fibers: }50$\mu m $ \mbox{}\protect\newline \textbf{Notes:} 15$m $ yield in 2$min $ &
  \unskip~\cite{527120:11974316}\\\cline{1-1}\cline{2-2}\cline{3-3}\cline{4-4}\cline{5-5}
Polystyrene (PS) &
  1,2,4-Trichloro benzene &
  Electrohydro-dynamic (EHD) jet printing &
  \textbf{Solution Concentration:} 1 to 5$wt\% $ PS \mbox{}\protect\newline \textbf{Nozzle:} Glass nozzle inner diameter of 2$\mu m $ and outer diameter of 2.66$\mu m $ \mbox{}\protect\newline \textbf{Solution deposition rate:} Si \mbox{}\protect\newline \textbf{Nozzle-to-substrate distance: 20, 30, 40$\mu m $} \mbox{}\protect\newline \textbf{Substrate composition: } \mbox{}\protect\newline \textbf{Applied voltage:} 500 to 400$V $ in 25$V $ increments \mbox{}\protect\newline \textbf{x-y stage velocity:} 0.01-10$mm/s $ \mbox{}\protect\newline \textbf{Fiber Diameter:} about 60-170$\mu m $ \mbox{}\protect\newline \textbf{Distance between adjacent fibers:} \textit{Not determined} &
  \unskip~\cite{527120:11974320}\\\cline{1-1}\cline{2-2}\cline{3-3}\cline{4-4}\cline{5-5}
Poly(ethylene oxide) (PEO) &
  \textit{Not determined} &
  Typical NFES process &
  \textbf{Solution Concentration:} 3$wt\% $ PEO \mbox{}\protect\newline \textbf{Nozzle:} \textit{Not determined} \mbox{}\protect\newline \textbf{Solution deposition rate:} \textit{Not determined} \mbox{}\protect\newline \textbf{Nozzle-to-substrate distance:} 500$\mu m $ \mbox{}\protect\newline \textbf{Substrate composition:} Si \mbox{}\protect\newline \textbf{Applied voltage:} 1000$V $ \mbox{}\protect\newline \textbf{x-y stage velocity:} 20$cm/s $ \mbox{}\protect\newline \textbf{Fiber Diameter:} 300$nm $ \mbox{}\protect\newline \textbf{Distance between adjacent fibers:} 25$\mu m $ &
  \unskip~\cite{527120:11974321}\\\cline{1-1}\cline{2-2}\cline{3-3}\cline{4-4}\cline{5-5}
Poly(ethylene oxide) (PEO) &
  Distilled water &
  Multinozzle NFES &
  \textbf{Solution Concentration:} 5$wt\% $ \mbox{}\protect\newline \textbf{Nozzle:} four-nozzle and six-nozzle array with needle spacing changes from 1.5$mm $ to 3.5$mm $ \mbox{}\protect\newline \textbf{Solution deposition rate:} 1-3$\mu L / min $ \mbox{}\protect\newline \textbf{Nozzle-to-substrate distance:} 2$mm $ \mbox{}\protect\newline \textbf{Substrate composition:} \textit{Not determined} \mbox{}\protect\newline \textbf{Applied voltage:} 1.7-2.7$kV $ \mbox{}\protect\newline \textbf{x-y stage velocity:} \textit{Not determined} \mbox{}\protect\newline \textbf{Fiber Diameter:} 5.47$\mu m $ \mbox{}\protect\newline \textbf{Distance between adjacent fibers:} 3-5 $mm $ &
  \unskip~\cite{527120:11974322}\\\cline{1-1}\cline{2-2}\cline{3-3}\cline{4-4}\cline{5-5}
Poly(ethylene oxide) (PEO) &
  Distilled water &
  Multinozzle NFES &
  \textbf{Solution Concentration:}5$wt\% $ \mbox{}\protect\newline \textbf{Nozzle:} Dual-28G-needle array with needle inner diameter of 0.18$mm $ and outer diameter of 0.36$mm $; with needle spacing changes from 2.0$mm $ to 3.0$mm $ \mbox{}\protect\newline \textbf{Solution deposition rate:} 0.2$\mu L / min $ \mbox{}\protect\newline \textbf{Nozzle-to-substrate distance:} 3.0-4.0$mm $ \mbox{}\protect\newline \textbf{Substrate composition: } \textit{Not determined} \mbox{}\protect\newline \textbf{Applied voltage:} 2.0-3.0$kV $ \mbox{}\protect\newline \textbf{x-y stage velocity:} 20$mm/s $ \mbox{}\protect\newline \textbf{Fiber Diameter:} \textit{Not determined} \mbox{}\protect\newline \textbf{Distance between adjacent fibers:} 218-326$\mu m $ &
  \unskip~\cite{527120:11974323}\\\cline{1-1}\cline{2-2}\cline{3-3}\cline{4-4}\cline{5-5}
Poly(ethylene oxide) (PEO) &
  Distilled water &
  Multinozzle NFES &
  \textbf{Solution Concentration:}5 $wt\% $ \mbox{}\protect\newline \textbf{Nozzle:} Dual-28G-needle array with needle inner diameter of 180$\mu m $ and outer diameter of 360$\mu m $; with needle spacing changes of 2.0$mm $ \mbox{}\protect\newline \textbf{Solution deposition rate:} 0.2$\mu L / min $ \mbox{}\protect\newline \textbf{Nozzle-to-substrate distance:} 4.0$mm $ \mbox{}\protect\newline \textbf{Substrate composition:} chromium-plated glass \mbox{}\protect\newline \textbf{Applied voltage:} 2.5$kV $ \mbox{}\protect\newline \textbf{x-y stage velocity:} 20$mm/s $ \mbox{}\protect\newline \textbf{Fiber Diameter:} \textit{Not determined} \mbox{}\protect\newline \textbf{Distance between adjacent fibers:} 2.3002-2.7224$mm $ &
  \unskip~\cite{527120:11974324}\\\cline{1-1}\cline{2-2}\cline{3-3}\cline{4-4}\cline{5-5}
Poly(ethylene oxide) (PEO) &
  \textit{Not determined} &
  Typical NFES process &
  \textbf{Solution Concentration:} 2$wt\% $ \mbox{}\protect\newline \textbf{Nozzle:} G30 needle with inner diameter of 0.15$mm $ \mbox{}\protect\newline \textbf{Solution deposition rate:} \textit{Not determined} \mbox{}\protect\newline \textbf{Nozzle-to-substrate distance:} 1-3$mm $ \mbox{}\protect\newline \textbf{Substrate composition:} Silicon \mbox{}\protect\newline \textbf{Applied voltage:} 1250$V $ \mbox{}\protect\newline \textbf{x-y stage velocity:} \textit{Not determined} \mbox{}\protect\newline \textbf{Fiber Diameter:} \textit{Not determined} \mbox{}\protect\newline \textbf{Distance between adjacent fibers:} 20$\mu m $ &
  \unskip~\cite{527120:11974325}\\\cline{1-1}\cline{2-2}\cline{3-3}\cline{4-4}\cline{5-5}
Gelatin \mbox{}\protect\newline (porcine skin) &
  Acetic Acid and Ethyl Acetate &
  Typical NFES process &
  \textbf{Solution Concentration:} 11$wt\% $ gelatin, 30$wt\% $ water, 35.4$wt\% $ acetic acid, 23.6$wt\% $ ethyl acetate \mbox{}\protect\newline \textbf{Nozzle:} 19G needle tip with outer diameter of 1.08$mm $ \mbox{}\protect\newline \textbf{Solution deposition rate:} \textit{Not determined} \mbox{}\protect\newline \textbf{Nozzle-to-substrate distance:} 1.25$mm $ \mbox{}\protect\newline \textbf{Substrate composition:} Poly(Dimethylsiloxane) (PDMS) films \mbox{}\protect\newline \textbf{Applied voltage:} 1000$V $ \mbox{}\protect\newline \textbf{x-y stage velocity:} \textit{Not determined} \mbox{}\protect\newline \textbf{Fiber Diameter:} around 2-3$\mu m $ \mbox{}\protect\newline \textbf{Distance between adjacent fibers:} 40$\mu m $ &
  \unskip~\cite{527120:11974326}\\\cline{1-1}\cline{2-2}\cline{3-3}\cline{4-4}\cline{5-5}
Poly(ethylene oxide) (PEO) &
  Water/Ethanol (v/v = 60/40) &
  Typical NFES process &
  \textbf{Solution Concentration:} PEO concentrations of 16\% adn 18\% \mbox{}\protect\newline \textbf{Nozzle:} 40$\mu m $ \mbox{}\protect\newline \textbf{Solution deposition rate:} \mbox{}\protect\newline \textbf{Nozzle-to-substrate distance:} 1$mm $ \mbox{}\protect\newline \textbf{Substrate composition:} Planar silicon \mbox{}\protect\newline \textbf{Applied voltage:} 1.7$kV $ \mbox{}\protect\newline \textbf{x-y stage velocity:} 0.36$m/s $ \mbox{}\protect\newline \textbf{Fiber Diameter:} 5.15$\mu m $ \mbox{}\protect\newline \textbf{Distance between adjacent fibers:} \textit{Not determined} &
  \unskip~\cite{527120:11974327}\\\cline{1-1}\cline{2-2}\cline{3-3}\cline{4-4}\cline{5-5}
Poly(ethylene oxide) (PEO) &
  Water/Ethanol (v/v = 3/1) &
  Electrohydro-dynamic Direct-Write (EDW) &
  \textbf{Solution Concentration:} 14$wt\% $ PEO \mbox{}\protect\newline \textbf{Nozzle:} Stainless needle with inner diameter of 210$\mu m $ and outer diameter of 400$\mu m $ \mbox{}\protect\newline \textbf{Solution deposition rate:} 50$\mu L/h $ \mbox{}\protect\newline \textbf{Nozzle-to-substrate distance:} 2$mm $ \mbox{}\protect\newline \textbf{Substrate composition:} Poly(ethylene terephthalate) (PET) \mbox{}\protect\newline \textbf{Applied voltage:} 3$kV $ \mbox{}\protect\newline \textbf{x-y stage velocity:} 700$mm/s $ \mbox{}\protect\newline \textbf{Fiber Diameter:} 15-35$\mu m $ \mbox{}\protect\newline \textbf{Distance between adjacent fibers:} 70$\mu m $ &
  \unskip~\cite{527120:11974328}\\\cline{1-1}\cline{2-2}\cline{3-3}\cline{4-4}\cline{5-5}
Poly(ethylene oxide) (PEO) &
  Deionized water &
  Mechano-Electrospinning &
  \textbf{Solution Concentration:} 3$wt\% $ PEO \mbox{}\protect\newline \textbf{Nozzle:} Stainless steel nozzle with inner diameter of 160$\mu m $ and outer diameter of 310$\mu m $ \mbox{}\protect\newline \textbf{Solution deposition rate:} 50$nL/min $ \mbox{}\protect\newline \textbf{Nozzle-to-substrate distance:} 2-5$mm $ \mbox{}\protect\newline \textbf{Substrate composition:} Silicone \mbox{}\protect\newline \textbf{Applied voltage:}polymer jet initiated at 2$kV $ and dispensed at 1$kV $ \mbox{}\protect\newline \textbf{x-y stage velocity:} 200-400$mm/s $ \mbox{}\protect\newline \textbf{Fiber Diameter:} from 344$\pm $32 to 214$\pm $27$nm $ \mbox{}\protect\newline \textbf{Distance between adjacent fibers:} \textit{Not determined} &
  \unskip~\cite{527120:11974304}\\\cline{1-1}\cline{2-2}\cline{3-3}\cline{4-4}\cline{5-5}
Poly(co-Glycolic) acid (PLGA)  &
  Dimethyl Carbonate (DMC) &
  Tethered Pyro-Electrohydro-dynamic Spinning (TPES) &
  \textbf{Solution Concentration:} \textit{Not determined} \mbox{}\protect\newline \textbf{Nozzle:} nozzle-free \mbox{}\protect\newline \textbf{Solution deposition rate:} The drop reservoir is placed directly on a flat substrate \mbox{}\protect\newline \textbf{Nozzle-to-substrate distance:} Taylor's cone is focused and put in direct contact with the collector \mbox{}\protect\newline \textbf{Substrate composition:} Poly(tetrafluoroethylene) (PTFE) coated glass slide \mbox{}\protect\newline \textbf{Applied voltage:} pyro-electric field of between 2.7 $x10^{7}\;V/m $ and 5.5$x10^{7}\;V/m $ \mbox{}\protect\newline \textbf{x-y stage velocity:} \textit{Not determined} \mbox{}\protect\newline \textbf{Fiber Diameter:} 304.7$nm $ \mbox{}\protect\newline \textbf{Distance between adjacent fibers:} \textit{Not determined} &
  \unskip~\cite{527120:11974307}\\\cline{1-1}\cline{2-2}\cline{3-3}\cline{4-4}\cline{5-5}
Poly(ethylene oxide) (PEO) with Tetrabutylammonium tetrafluoroborate (TBF) and SU-8 2002 &
  N,N Dimethylformamide (DMF) &
  Typical NFES process &
  \textbf{Solution Concentration:} SU-8/PEO/TBF blend with 0.75$wt\% $ PEO, 1$wt\% $ TBF; the blend is diluted with 30$vol\% $ DMF \mbox{}\protect\newline $\mu m $$\mu m $ \mbox{}\protect\newline \textbf{Solution deposition rate:} \textit{Not determined} \mbox{}\protect\newline \textbf{Nozzle-to-substrate distance:} \textit{Not determined} \mbox{}\protect\newline \textbf{Substrate composition:} Brass disk with a diameter of 38$mm $ \mbox{}\protect\newline \textbf{Applied voltage:} 980$V $ \mbox{}\protect\newline \textbf{x-y stage velocity:} \textit{Not determined} \mbox{}\protect\newline \textbf{Fiber Diameter:} \textit{Not determined} \mbox{}\protect\newline \textbf{Distance between adjacent fibers:} \textit{Not determined} &
  \unskip~\cite{527120:12033655}\\\cline{1-1}\cline{2-2}\cline{3-3}\cline{4-4}\cline{5-5}
Poly(ethylene oxide) (PEO) &
  Water:Ethanol (3:2) &
  Suspension NFES &
  \textbf{Solution Concentration:} 14$wt\% $ PEO \mbox{}\protect\newline \textbf{Nozzle:} stainless steel needle (25 G) with inner diameter of 0.25$mm $ \mbox{}\protect\newline \textbf{Solution deposition rate:} 3$nL/s $ \mbox{}\protect\newline \textbf{Nozzle-to-substrate distance:} between 0.5 and 10$mm $ with 0.5$mm $ increments \mbox{}\protect\newline \textbf{Substrate composition:} Planar silicon electrodes \mbox{}\protect\newline \textbf{Applied voltage:} 1.6$kV $ \mbox{}\protect\newline \textbf{x-y stage velocity:} 50, 150, and 250$mm/s $ \mbox{}\protect\newline \textbf{Fiber Diameter:} 300$nm $ \mbox{}\protect\newline \textbf{Distance between adjacent fibers:} 0.1 and 0.5$mm $ &
  \unskip~\cite{527120:12033656}\\\cline{1-1}\cline{2-2}\cline{3-3}\cline{4-4}\cline{5-5}
Poly(ethylene oxide) (PEO) &
  Deionized water &
  Typical NFES process &
  \textbf{Solution Concentration:} 10$wt\% $ PEO \mbox{}\protect\newline \textbf{Nozzle:} 32G metal needle \mbox{}\protect\newline \textbf{Solution deposition rate:} (Jet impact speed of 5$mm/s $ ) \mbox{}\protect\newline \textbf{Nozzle-to-substrate distance:} 0.5$mm $ \mbox{}\protect\newline \textbf{Substrate composition:} p-type silicon wafer \mbox{}\protect\newline \textbf{Applied voltage:} 400$V $ \mbox{}\protect\newline \textbf{x-y stage velocity:} 5$mm/s $ \mbox{}\protect\newline \textbf{Fiber Diameter:} \mbox{}\protect\newline \textbf{Distance between adjacent fibers:} 50$\mu m $ &
  \unskip~\cite{527120:12033657}\\
\tblbottomrule 
\end{longtable}
\endgroup
\makeatletter\@ifundefined{TwoColDocument}{}{\twocolumn}\makeatother 
\end{landscape}

    
\section{NFES Variants}
[SECTION UNDERWORK]



\subsection{Low-Voltage NFES (LV NFES) \unskip~\protect\cite{527120:11973130}}Some differences have been discovered between LV-NFES and conventional NFES. Low voltage near field electrospinning produces thinner fibers with lower voltages. Moreover, when implementing a moving stage, the fibers are affected by the mechanical stretching. Bisht et al. (2011) reported that thinner diameters are yield with the increase of the x-y stage velocity, and larger diameters by decreasing the stage velocity.



\subsection{Scanning Tip Electrospinning \unskip~\protect\cite{527120:11974306}}Lorem ipsum dolor sit amet, consectetur adipiscing elit, sed do eiusmod tempor incididunt ut labore et dolore magna aliqua.



\subsection{3D Electrospinning \unskip~\protect\cite{527120:11974313} \mbox{}\protect\newline Electrohydro-dynamic 3D Print-patterning or Electrohydro-dynamic Jetting \unskip~\protect\cite{527120:11974310}}Lorem ipsum dolor sit amet, consectetur adipiscing elit, sed do eiusmod tempor incididunt ut labore et dolore magna aliqua.



\subsection{Multinozzle NFES \unskip~\protect\cite{527120:11974322,527120:11974323,527120:11974324}}Lorem ipsum dolor sit amet, consectetur adipiscing elit, sed do eiusmod tempor incididunt ut labore et dolore magna aliqua.



\subsection{Electrohydro-dynamic Writing or Mechanoelectrospinning (MES) \unskip~\protect\cite{527120:11974311} \mbox{}\protect\newline Electrohydro-dynamic Direct-Write (EDW) \unskip~\protect\cite{527120:11974328} \mbox{}\protect\newline Mechano-Electrospinning \unskip~\protect\cite{527120:11974304}}Lorem ipsum dolor sit amet, consectetur adipiscing elit, sed do eiusmod tempor incididunt ut labore et dolore magna aliqua.



\subsection{Suspension NFES \unskip~\protect\cite{527120:12033656}}Lorem ipsum dolor sit amet, consectetur adipiscing elit, sed do eiusmod tempor incididunt ut labore et dolore magna aliqua.



\subsection{Helix Electrohydro-dynamic Printing (HE-printing) \unskip~\protect\cite{527120:11974308} \mbox{}\protect\newline Electrohydro-dynamic (EHD) jet printing \unskip~\protect\cite{527120:11974320}}Lorem ipsum dolor sit amet, consectetur adipiscing elit, sed do eiusmod tempor incididunt ut labore et dolore magna aliqua.



\subsection{Airflow-assisted Electrohydro-dynamic Direct-writing (EDW) \unskip~\protect\cite{527120:11974312}}Lorem ipsum dolor sit amet, consectetur adipiscing elit, sed do eiusmod tempor incididunt ut labore et dolore magna aliqua.



\subsection{Tethered Pyro-Electrohydro-dynamic Spinning (TPES) \unskip~\protect\cite{527120:11974307}}Lorem ipsum dolor sit amet, consectetur adipiscing elit, sed do eiusmod tempor incididunt ut labore et dolore magna aliqua. 
    
\section{Conclusion}
Lorem ipsum dolor sit amet, consectetur adipiscing elit, sed do eiusmod tempor incididunt ut labore et dolore magna aliqua. Ut enim ad minim veniam, quis nostrud exercitation ullamco laboris nisi ut aliquip ex ea commodo consequat. Duis aute irure dolor in reprehenderit in voluptate velit esse cillum dolore eu fugiat nulla pariatur. Excepteur sint occaecat cupidatat non proident, sunt in culpa qui officia deserunt mollit anim id est laborum.


    



\bibliographystyle{elsarticle-num}

\bibliography{\jobname}

\end{document}
