% Chapter Template

%\begin{landscape}
%\pagebreak[4]\global\pdfpageattr\expandafter{\the\pdfpageattr/Rotate 90}

\chapter{Summary} % Main chapter title

\label{Chapter:Summary}

%\subsubsection*{\color{mygray}[Chapter under work]}
%Once the Methodology has been established, it is important to set up the corresponding activities in an organized and timely manner. This is done using a Workplan Schedule. This provides us a clear idea of the amount of work needed expressed in periods of time.

%----------------------------------------------------------------------%
\begin{table}[th]
\caption{Electrospun Polymer Solutions - Solution and Process Parameters}
\begin{center}
% total table width eq 0.91\textheight = 100%
\begin{tabular}{
>{\raggedright\arraybackslash}p{0.28\textwidth}
>{\raggedright\arraybackslash}p{0.72\textwidth} }

\hline
Polymer(s): &
Poly(ethylene oxide) (PEO) \\

\arrayrulecolor{lightgray}\hline
Solvent(s): &
Deionized water \\

\hline
NFES Variant: &
Low-Voltage NFES \\

\hline
Polymer Solution and Process Properties: &
\begin{itemize}[leftmargin=*]
\item PEO Concentration: 1, 2, and 3 $w t \%$
\item Rise in solution conductivity with the increase in PEO concentration
\item Solution Stirring: 24 $h$ of free diffusion followed by 96 $h$ of stirring at 30 $r p m$
\item 3 $m L$ syringe
\item 27 gauge type 304 stainless steel needle
\item Solution deposition rate: lower than 1 $\mu L / h$
\item needle-to-collector distance: 1 $m m$
\item Collector substrate: Pyrolyzed SU-8 carbon and Si
\item NFES process initiated by an air interference with a glass microprobe tip (1 to 3 $\mu m$ tip diameter) to overcome the surface tension
\item Time to produce a stable continuous jet: 45 $min$
\item Polymer jet initiated at 400-600 $V$ and dispensed at 200-400 $V$
\item Collector linear speed: 10-40 $m m / s$
\item The voltage turned on when the solution formed a full-sized droplet of 500 $\mu m$ diameter at the needle tip.
\end{itemize} \\

\hline
Fiber Characterization: &
\begin{itemize}[leftmargin=*]
\item Diameter: 50-425 $n m$
\end{itemize} \\

\hline
Ref: & \cite{Bisht2011} \\ %Controlled Continuous Patterning of Polymeric Nanofibers on Three-Dimensional Substrates Using Low-Voltage Near-Field Electrospinning
\arrayrulecolor{black}\hline
\label{tbl:FloresCompare}
\end{tabular}
\end{center}
\end{table}
%----------------------------------------------------------------------%

%----------------------------------------------------------------------%
\begin{table}[th]
\caption{Electrospun Polymer Solutions - Solution and Process Parameters}
\begin{center}
% total table width eq 0.91\textheight = 100%
\begin{tabular}{
>{\raggedright\arraybackslash}p{0.28\textwidth}
>{\raggedright\arraybackslash}p{0.72\textwidth} }

\hline
Polymer(s): &
Poly[2-methoxy-5-(2-ethylhexyloxy)-1,4-phenylenevinylene] (MEH-PPV) with Poly(ethylene oxide) (PEO) \\

\arrayrulecolor{lightgray}\hline
Solvent(s): &
acetonitrile / toluene mixture (65 / 35); acetic acid / toluene (17 / 83); pure toluene \\

\hline
NFES Variant: &
\emph{Not determined.} \\

\hline
Polymer Solution and Process Properties: &
\begin{itemize}[leftmargin=*]
\item Concentrations:
    \begin{itemize}[leftmargin=*]
    \item MEH-PPV solution: 10 $m g$ of MEH-PPV in 2 $m L$ of toluene
    \item 500 $\mu L$ of MEH-PPV solution with 250 $m g$ of PEO in 3.5 $m L$ of acetonitrile / toluene (65 / 35)
    \item 500 $\mu L$ of MEH-PPV solution with 250 $m g$ of PEO in 3 $m L$ of acetic acid / toluene (17 / 83)
    \item The resulting MEH-PPV/PEO concentration is 1:100
    \end{itemize}
\item Solution Stirring: MEH-PPV solution stirred for 4 $h$; PEO solution stirred for 8 $h$; MEH-PPV/PEO solution stirred and ultrasonically agitated
\item Collector substrate: SiO2/Si (oxide thickness = 800 $n m$)
\item needle-to-collector distance: 500 $\mu m$
\item $\mu m$-diameter tip Tungsten spinneret in a 26 gauge needle
\item Solution deposition rate: 50 $\mu L / h$
\item Electrostatic voltage: around 1.3 $k V$
\item x-y stage velocity: 50 $c m / s$
\end{itemize} \\

\hline
Fiber Characterization: &
\begin{itemize}[leftmargin=*]
\item Distance between adjacent fibers: around 100 $\mu m$
\item Fiber diameter: around 100 $n m$
\end{itemize} \\

\hline
Ref: & \cite{Camillo2013} \\ %Near-field electrospinning of conjugated polymer light-emitting nanofibers
\arrayrulecolor{black}\hline
\label{tbl:FloresCompare}
\end{tabular}
\end{center}
\end{table}
%----------------------------------------------------------------------%

%----------------------------------------------------------------------%
\begin{table}[th]
\caption{Electrospun Polymer Solutions - Solution and Process Parameters}
\begin{center}
% total table width eq 0.91\textheight = 100%
\begin{tabular}{
>{\raggedright\arraybackslash}p{0.28\textwidth}
>{\raggedright\arraybackslash}p{0.72\textwidth} }

\hline
Polymer(s): &
Poly(ethylene oxide) (PEO) \\

\arrayrulecolor{lightgray}\hline
Solvent(s): &
Water \\

\hline
NFES Variant: &
Scanning Tip Electrospinning and NFES \\

\hline
Polymer Solution and Process Properties: &
\begin{itemize}[leftmargin=*]
\item 7 wt \% PEO aqueous solution
\item Under room temperature at 1 $atm$
\item needle-to-collector distance: 500 $\mu m$
\item needle diameter: outer: 200 $\mu m$; inner: 100 $\mu m$
\item applied voltage for jet initiation: 1.5 $k V$
\item applied voltage for fiber deposition: 600 $V$
\item Mechanical drawing is applied by using a tungsten probe with 1 $\mu m$ tip diameter to poke inside the meniscus.
\item The probe is then rapidly pulled away from the polymer droplet to activate the continuous electrospinning process
\item polymer jet diameter: 3 $\mu m$
\item polymer feed rate: 0.1 $\mu L / h$
\item x-y stage velocity: 120 $m m / s$
\end{itemize} \\

\hline
Fiber Characterization: &
\begin{itemize}[leftmargin=*]
\item 108 $m$ yield in 15 $min$ with a fiber diameter of 709 $\pm$ 131 $n m$
\item Fiber diameter: around 49-74 $n m$ when applied voltage is 800 $V$
\end{itemize} \\

\hline
Ref: & \cite{Chang2008} \\ %Continuous near-field electrospinning for large area deposition of orderly nanofiber patterns
\arrayrulecolor{black}\hline
\label{tbl:FloresCompare}
\end{tabular}
\end{center}
\end{table}
%----------------------------------------------------------------------%

%----------------------------------------------------------------------%
\begin{table}[th]
\caption{Electrospun Polymer Solutions - Solution and Process Parameters}
\begin{center}
% total table width eq 0.91\textheight = 100%
\begin{tabular}{
>{\raggedright\arraybackslash}p{0.28\textwidth}
>{\raggedright\arraybackslash}p{0.72\textwidth} }

\hline
Polymer(s): &
Poly($\varepsilon$-Caprolactone) (PCL) \\

\arrayrulecolor{lightgray}\hline
Solvent(s): &
\emph{Not applicable.} \\

\hline
NFES Variant: &
Melt Electrospinning Writing (MEW) \\

\hline
Polymer Solution and Process Properties: &
\begin{itemize}[leftmargin=*]
\item Collector substrate: NCO-sP(EO-stat-PO)-coated glass slide surfaces
\item Accelerating voltage 2.0–10.0 $k V$
\item Collector distance: 1–10 $m m$
\item Heating temperature: 80–120 $^\circ C$
\item Feeding air pressure 0.5–4.0 $bar$
\item Spinneret diameters: 21, 23, 25, 27, 30, and 33 $G$
\item Axis velocity: 1000–9000 $m m / min$
\item Fibre spacing: 100 $\mu m$
\end{itemize} \\

\hline
Fiber Characterization: &
\begin{itemize}[leftmargin=*]
\item Filament surface is smooth and homogeneous
\item The crystalline regions formed perpendicular to the filament
\item Fiber diameter: 817 $\pm$ 165 $n m$
\end{itemize} \\

\hline
Ref: & \cite{Dalton2015} \\ % Additive manufacturing of scaffolds with sub-micron filaments via melt electrospinning writing Related content Patterned melt electrospun substrates for tissue engineering
\arrayrulecolor{black}\hline
\label{tbl:FloresCompare}
\end{tabular}
\end{center}
\end{table}
%----------------------------------------------------------------------%

%----------------------------------------------------------------------%
\begin{table}[th]
\caption{Electrospun Polymer Solutions - Solution and Process Parameters}
\begin{center}
% total table width eq 0.91\textheight = 100%
\begin{tabular}{
>{\raggedright\arraybackslash}p{0.28\textwidth}
>{\raggedright\arraybackslash}p{0.72\textwidth} }

\hline
Polymer(s): &
Poly(vinylidine fluorid) (PVDF) \\

\arrayrulecolor{lightgray}\hline
Solvent(s): &
N,N-dimethylformamide (DMF) \\

\hline
NFES Variant: &
Helix Electrohydrodynamic Printing (HE-printing) \\

\hline
Polymer Solution and Process Properties: &
\begin{itemize}[leftmargin=*]
\item 1.8 $g$ PVDF in 4.1 $g$ of DMF and 4.1 $g$ of acetone to obtain a concentration of 18\%
\item Solution kept at 35 $^\circ C$ for about 6 $h$ until the solution was homogeneous.
\item Collector substrate: Poly(dimethylsiloxane) (PDMS) on Ecoflex
\item Solution feed rate: 400 $n L / min$
\item Needle diameter: inner 260 $\mu m$; external 510 $\mu m$
\item Applied voltage: 1.5–3 $k V$
\item Nozzle-to-collector distance: 10-50 $m m$
\item x-y stage velocity: 0-400 $m m / min$
\item At room temperature and 35–45\% humidity
\end{itemize} \\

\hline
Fiber Characterization: &
\begin{itemize}[leftmargin=*]
\item Stretchable serpentine structures with specific wavelength and amplitude.
\item Wavelength: about 100-2000 $\mu m$
\item Fiber diameter: about 1.5-3 $\mu m$
\end{itemize} \\

\hline
Ref: & \cite{Duan2017} \\ %Helix Electrohydrodynamic Printing of Highly Aligned Serpentine Micro/Nanofibers.
\arrayrulecolor{black}\hline
\label{tbl:FloresCompare}
\end{tabular}
\end{center}
\end{table}
%----------------------------------------------------------------------%

%----------------------------------------------------------------------%
\begin{table}[th]
\caption{Electrospun Polymer Solutions - Solution and Process Parameters}
\begin{center}
% total table width eq 0.91\textheight = 100%
\begin{tabular}{
>{\raggedright\arraybackslash}p{0.28\textwidth}
>{\raggedright\arraybackslash}p{0.72\textwidth} }

\hline
Polymer(s): &
Polyhedral Oligomeric Silsesquioxane-Poly(Carbonate-Urea)Urethane (POSS-PCU) and Polyhedral Oligomeric Silsesquioxane-Poly(Caprolactone-Poly(Carbonate-Urea)Urethane) (POSS-PCL-PCU)\\

\arrayrulecolor{lightgray}\hline
Solvent(s): &
Dimethylacetamide (DMAC) and 1-Butanol \\

\hline
NFES Variant: &
Electrohydrodynamic 3D Print-patterning or Electrohydrodynamic Jetting \\

\hline
Polymer Solution and Process Properties: &
\begin{itemize}[leftmargin=*]
\item Solution concentration: POSS-PCU and POSS-PCL-PCU used in 20\% w/w concentration in DMAC
\item Needle diameter: 750 $\mu m$
\item Applied voltage: 8.0-10.0 $k V$
\item Solution flow rate: less than 1 $\mu L / min$
\item Needle-to-collector distance: about between 500 $\mu m$ to 2 $m m$
\item x-y stage velocity: 10 $m m / s$
\item Ethanol-coated substrate
\end{itemize} \\

\hline
Fiber Characterization: &
\begin{itemize}[leftmargin=*]
\item Distance between adjacent fibers: 250 $\mu m$
\item Fiber diameter: 5-50 $\mu m$
\end{itemize} \\

\hline
Ref: & \cite{Gupta2007} \\ %Novel Electrohydrodynamic Printing of Nanocomposite Biopolymer Scaffolds
\arrayrulecolor{black}\hline
\label{tbl:FloresCompare}
\end{tabular}
\end{center}
\end{table}
%----------------------------------------------------------------------%

%----------------------------------------------------------------------%
\begin{table}[th]
\caption{Electrospun Polymer Solutions - Solution and Process Parameters}
\begin{center}
% total table width eq 0.91\textheight = 100%
\begin{tabular}{
>{\raggedright\arraybackslash}p{0.28\textwidth}
>{\raggedright\arraybackslash}p{0.72\textwidth} }

\hline
Polymer(s): &
Poly(ethylene oxide) (PEO) \\

\arrayrulecolor{lightgray}\hline
Solvent(s): &
Distilled water \\

\hline
NFES Variant: &
Electrohydrodynamic Writing or Mechanoelectrospinning (MES) \\

\hline
Polymer Solution and Process Properties: &
\begin{itemize}[leftmargin=*]
\item Polymer solution weight concentration: 6 $w t \%$ PEO
\item Needle-to-collector distance: 7.5 $m m$
\item Applied voltage to initiate the jet: 2 $k V$
\item Applied voltage during deposition: 0.8-1 $k V$
\item Under the room temperature and relative humidity of about 25\%.
\item x-y stage velocity: around 400 $m m / s$
\item Solution flow rate: 1200 $n L / min$
\end{itemize} \\

\hline
Fiber Characterization: &
\begin{itemize}[leftmargin=*]
\item Distance between adjacent fibers: 5 $\mu m$
\item Fiber diameter: 200-350 $n m$
\end{itemize} \\

\hline
Ref: & \cite{Huang2015} \\ %Versatile, kinetically controlled, high precision electrohydrodynamic writing of micro/nanofibers
\arrayrulecolor{black}\hline
\label{tbl:FloresCompare}
\end{tabular}
\end{center}
\end{table}
%----------------------------------------------------------------------%

%----------------------------------------------------------------------%
\begin{table}[th]
\caption{Electrospun Polymer Solutions - Solution and Process Parameters}
\begin{center}
% total table width eq 0.91\textheight = 100%
\begin{tabular}{
>{\raggedright\arraybackslash}p{0.28\textwidth}
>{\raggedright\arraybackslash}p{0.72\textwidth} }

\hline
Polymer(s): &
Poly(ethylene oxide) (PEO) \\

\arrayrulecolor{lightgray}\hline
Solvent(s): &
Deionized water and the ethanol with a volume ratio of 3:1 \\

\hline
NFES Variant: &
Airflow-assisted Electrohydrodynamic Direct-writing (EDW) \\

\hline
Polymer Solution and Process Properties: &
\begin{itemize}[leftmargin=*]
\item Concentration: 8 $w t \%$ PEO
\item Outer airflow passage diameter: 1 $m m$
\item Airflow gas pump pressure: 25 $k Pa$
\item Inner liquid passage diameter: 0.21 $m m$
\item Silicon substrate
\item Needle-to-collector distance: 2 $m m$
\item Solution flow rate: 30 $\mu L / h$
\item Applied voltage: about 2 $k V$
\item x-y stage velocity: between 1-20 $m m / s$
\end{itemize} \\

\hline
Fiber Characterization: &
\begin{itemize}[leftmargin=*]
\item Fiber deposition position accuracy: 5.13 $\pm$ 6.67 $\mu m$
\item Fiber diameter: 3.73 $\pm$ 1.37 $\mu m$
\end{itemize} \\

\hline
Ref: & \cite{Jiang2018} \\ %Electrohydrodynamic Direct-Writing Micropatterns with Assisted Airflow
\arrayrulecolor{black}\hline
\label{tbl:FloresCompare}
\end{tabular}
\end{center}
\end{table}
%----------------------------------------------------------------------%

%----------------------------------------------------------------------%
\begin{table}[th]
\caption{Electrospun Polymer Solutions - Solution and Process Parameters}
\begin{center}
% total table width eq 0.91\textheight = 100%
\begin{tabular}{
>{\raggedright\arraybackslash}p{0.28\textwidth}
>{\raggedright\arraybackslash}p{0.72\textwidth} }

\hline
Polymer(s): &
Poly(Vinylidene Fluoride) (PVDF) \\

\arrayrulecolor{lightgray}\hline
Solvent(s): &
Acetone and Dimethyl Sulfoxide (DMSO) \\

\hline
NFES Variant: &
3D Electrospinning \\

\hline
Polymer Solution and Process Properties: &
\begin{itemize}[leftmargin=*]
\item Capstone FS-66 used as an anionic surfactant.
\item Solution concentration: 17 $w t \%$ PVDF
    \begin{itemize}[leftmargin=*]
    \item 1.7 $g$ of PVDF added to 5 $g$ of acetone and dispersed for 30 $min$
    \item 0.5 $g$ of Capstone FS-66 was added to 5 $g$ of DMSO and dispersed
    \item then both solutions are mixed for more than 1 $h$
    \end{itemize}
\item Collector substrate: A4 size commercial printing paper (Double A)
\item Needle inner diameter: 100 $\mu m$
\item x-y stage velocity: 10 $m m / s$
\item Solution flow rate: 14 $n L / min$
\item Needle-to-collector distance: 750 $\mu m$
\item Applied voltage: 1.9 $k V$
\end{itemize} \\

\hline
Fiber Characterization: &
\begin{itemize}[leftmargin=*]
\item A stack of fibers was produced, but fiber diameter is not reported.
\end{itemize} \\

\hline
Ref: & \cite{Kim2018} \\ %Characterization of 3D electrospinning on inkjet printed conductive pattern on paper
\arrayrulecolor{black}\hline
\label{tbl:FloresCompare}
\end{tabular}
\end{center}
\end{table}
%----------------------------------------------------------------------%

%----------------------------------------------------------------------%
\begin{table}[th]
\caption{Electrospun Polymer Solutions - Solution and Process Parameters}
\begin{center}
% total table width eq 0.91\textheight = 100%
\begin{tabular}{
>{\raggedright\arraybackslash}p{0.28\textwidth}
>{\raggedright\arraybackslash}p{0.72\textwidth} }

\hline
Polymer(s): &
Polycaprolactone \\

\arrayrulecolor{lightgray}\hline
Solvent(s): &
Chloroform \\

\hline
NFES Variant: &
Direct-Write Electrospinning \\

\hline
Polymer Solution and Process Properties: &
\begin{itemize}[leftmargin=*]
\item Needle inner diameter: 200 $\mu m$
\item Ground electrode to collector plate distance: 50 $\mu m$
\item Solution concentration: 8.8 $w t \%$ Polycaprolactone
\item Solution stirred for 120 $min$
\item Applied voltage: 25 $k V$
\item Needle-to-collector distance: 70 $m m$
\item Solution flow rate: 0.1 $m L / h$
\item Electrospun at 23 $^\circ C$ and relative humidity between 54 and 57 \%
\item x-y stage velocity: 2-200 $m m / s$
\end{itemize} \\

\hline
Fiber Characterization: &
\begin{itemize}[leftmargin=*]
\item Fiber diameter: 400 $n m$ to 950 $\mu m$
\end{itemize} \\

\hline
Ref: & \cite{Lee2012} \\ %Fabrication of Patterned Nanofibrous Mats Using Direct-Write Electrospinning
\arrayrulecolor{black}\hline
\label{tbl:FloresCompare}
\end{tabular}
\end{center}
\end{table}
%----------------------------------------------------------------------%

%----------------------------------------------------------------------%
\begin{table}[th]
\caption{Electrospun Polymer Solutions - Solution and Process Parameters}
\begin{center}
% total table width eq 0.91\textheight = 100%
\begin{tabular}{
>{\raggedright\arraybackslash}p{0.28\textwidth}
>{\raggedright\arraybackslash}p{0.72\textwidth} }

\hline
Polymer(s): &
Poly(Vinylidene Fluoride) (PVDF) \\

\arrayrulecolor{lightgray}\hline
Solvent(s): &
Dimethyl Sulfoxide (DMSO)
    \begin{itemize}[leftmargin=*]
    \item Acetone and surfactant (ZONYL UR) were applied to improve the evaporation rate and to reduce the surface tension.
    \end{itemize} \\

\hline
NFES Variant: &
Hollow Cylindrical Near-Field Electrospinning (HCNFES) \\

\hline
Polymer Solution and Process Properties: &
\begin{itemize}[leftmargin=*]
\item Solution concentration: 18 $w t \%$ PVDF
    \begin{itemize}[leftmargin=*]
    \item DMSO:acetone concentration is 1:1
    \item Surfactant amount: 0.2 $g$
    \end{itemize}
\item Solution preparation:
    \begin{itemize}[leftmargin=*]
    \item PVDF-acetone stirred for 30 $min$
    \item sufractant-DMSO stirred for 30 $min$
    \item Both solutions stirred for 60 $min$
    \item Placed in a vacuum chamber for 15 $min$ to remove bubbles
    \end{itemize}
\item Needle-to-collector distance: 0.5 $m m$
\item Applied voltage: 14 $k V$
\item Tube collector rotational velocity: 1900 $rpm$
    \begin{itemize}[leftmargin=*]
    \item Tangential speed: 1989.3 $m m / s$
    \end{itemize}
\item Collector substrate: Poly(ethylene terephthalate) (PET)
\end{itemize} \\

\hline
Fiber Characterization: &
\begin{itemize}[leftmargin=*]
\item Fiber diameter: around 1.2 $\mu m$ 
\end{itemize} \\

\hline
Ref: & \cite{Liu2014} \\ %Direct-write PVDF nonwoven fiber fabric energy harvesters via the hollow cylindrical near-field electrospinning process
\arrayrulecolor{black}\hline
\label{tbl:FloresCompare}
\end{tabular}
\end{center}
\end{table}
%----------------------------------------------------------------------%

%----------------------------------------------------------------------%
\begin{table}[th]
\caption{Electrospun Polymer Solutions - Solution and Process Parameters}
\begin{center}
% total table width eq 0.91\textheight = 100%
\begin{tabular}{
>{\raggedright\arraybackslash}p{0.28\textwidth}
>{\raggedright\arraybackslash}p{0.72\textwidth} }

\hline
Polymer(s): &
 \\

\arrayrulecolor{lightgray}\hline
Solvent(s): &
Electrohydrodynamic Organic Nanowire printing (ONP) \\

\hline
NFES Variant: &
 \\

\hline
Polymer Solution and Process Properties: &
\begin{itemize}[leftmargin=*]
\item 1
\end{itemize} \\

\hline
Fiber Characterization: &
\begin{itemize}[leftmargin=*]
\item a
\end{itemize} \\

\hline
Ref: & \cite{Min2013} \\ %Large-scale organic nanowire lithography and electronics
\arrayrulecolor{black}\hline
\label{tbl:FloresCompare}
\end{tabular}
\end{center}
\end{table}
%----------------------------------------------------------------------%

%\cite{Bisht2011}   \\ %Controlled Continuous Patterning of Polymeric Nanofibers on Three-Dimensional Substrates Using Low-Voltage Near-Field Electrospinning
%\cite{Camillo2013} \\ %Near-field electrospinning of conjugated polymer light-emitting nanofibers
%\cite{Chang2008}   \\ %Continuous near-field electrospinning for large area deposition of orderly nanofiber patterns
%\cite{Dalton2015}  \\ % Additive manufacturing of scaffolds with sub-micron filaments via melt electrospinning writing Related content Patterned melt electrospun substrates for tissue engineering
%\cite{Duan2017}    \\ %Helix Electrohydrodynamic Printing of Highly Aligned Serpentine Micro/Nanofibers.
%\cite{Gupta2007}   \\ %Novel Electrohydrodynamic Printing of Nanocomposite Biopolymer Scaffolds
%\cite{Huang2015}   \\ %Versatile, kinetically controlled, high precision electrohydrodynamic writing of micro/nanofibers
%\cite{Jiang2018}   \\ %Electrohydrodynamic Direct-Writing Micropatterns with Assisted Airflow
%\cite{Kim2018}     \\ %Characterization of 3D electrospinning on inkjet printed conductive pattern on paper
%\cite{Lee2012}     \\ %Fabrication of Patterned Nanofibrous Mats Using Direct-Write Electrospinning
%\cite{Liu2014}     \\ %Direct-write PVDF nonwoven fiber fabric energy harvesters via the hollow cylindrical near-field electrospinning process
%\cite{Min2013}     \\ %Large-scale organic nanowire lithography and electronics
\cite{Pan2014}     \\ %Poly($\gamma$-benzyl $\alpha$, l-glutamate) in Cylindrical Near-Field Electrospinning Fabrication and Analysis of Piezoelectric Fibers
\cite{Pan2015}     \\ %Near-field electrospinning enhances the energy harvesting of hollow PVDF piezoelectric fibers
\cite{Song2015}    \\ %Patterned polydiacetylene-embedded polystyrene nanofibers based on electrohydrodynamic jet printing
\cite{Sun2006}     \\ %Near-Field Electrospinning
\cite{Wang2015}    \\ %Research on Multinozzle Near-Field Electrospinning Patterned Deposition
\cite{Wang2017}    \\ %Controllable deposition distance of aligned pattern via dual-nozzle near-field electrospinning
\cite{Wang2018}    \\ %Fabrication and evaluation of controllable deposition distance for aligned pattern by multi-nozzle near-field electrospinning
\cite{Xu2014}      \\ %Accuracy Improvement of Nano-fiber Deposition by Near-Field Electrospinning
\cite{Xue2014}     \\ %Rapid Patterning of 1-D Collagenous Topography as an ECM Protein Fibril Platform for Image Cytometry
\cite{Zheng2010}   \\ %Precision deposition of a nanofibre by near-field electrospinning
\cite{Zheng2014}   \\ %Electrohydrodynamic Direct-Write Orderly Micro/Nanofibrous Structure on Flexible Insulating Substrate
\cite{Zheng2012}   \\ %Polymer nanofibers prepared by low-voltage

%-%-%-%-%-%-%-%--------------------------------------------------------%
\begin{table}[th]
\caption{Electrospun Polymer Solutions - Solution and Process Parameters}
\begin{center}
% total table width eq 0.91\textheight = 100%
\begin{tabular}{
>{\raggedright\arraybackslash}p{0.28\textwidth}
>{\raggedright\arraybackslash}p{0.72\textwidth} }

\hline
Polymer(s): &
 \\

\arrayrulecolor{lightgray}\hline
Solvent(s): &
 \\

\hline
NFES Variant: &
 \\

\hline
Polymer Solution and Process Properties: &
\begin{itemize}[leftmargin=*]
\item 1
\end{itemize} \\

\hline
Fiber Characterization: &
\begin{itemize}[leftmargin=*]
\item a
\end{itemize} \\

\hline
Ref: & \cite{Bisht2011} \\
\arrayrulecolor{black}\hline
\label{tbl:FloresCompare}
\end{tabular}
\end{center}
\end{table}
%----------------------------------------------------------------------%

%----------------------------------------------------------------------------------------
%	SECTION 1
%----------------------------------------------------------------------------------------



%-----------------------------------
%	SUBSECTION 1
%-----------------------------------


