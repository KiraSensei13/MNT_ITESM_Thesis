% Chapter Template

\begin{landscape}
\pagebreak[4]\global\pdfpageattr\expandafter{\the\pdfpageattr/Rotate 90}

\chapter{Summary} % Main chapter title

\label{Chapter:Summary}

%\subsubsection*{\color{mygray}[Chapter under work]}
%Once the Methodology has been established, it is important to set up the corresponding activities in an organized and timely manner. This is done using a Workplan Schedule. This provides us a clear idea of the amount of work needed expressed in periods of time.

% move margin to fit table
%\addtolength{\oddsidemargin}{-1.5in}
%\addtolength{\evensidemargin}{-.875in}
\addtolength{\topmargin}{0.5in}

\begin{table}[th]
\caption{Electrospun Polymer Solutions - Solution and Process Parameters}
%\begin{center}
% total table width eq 0.91\textheight = 100%
\begin{tabular}{
>{\raggedright\arraybackslash}p{0.130\textheight}
>{\raggedright\arraybackslash}p{0.130\textheight}
>{\raggedright\arraybackslash}p{0.090\textheight}
>{\raggedright\arraybackslash}p{0.350\textheight}
>{\raggedright\arraybackslash}p{0.180\textheight}
>{\raggedright\arraybackslash}p{0.020\textheight} }  
\hline
Polymer(s) & Solvent(s) & NFES Variant & Polymer Solution and Process Properties & Fiber Characterization & Ref. \\
\hline
Poly(ethylene oxide) (PEO) &
Deionized water &
Mechano electrospinning &
\begin{itemize}[leftmargin=*]
\item PEO Concentration: 1, 2, and 3 $w t \%$
\item Rise in solution conductivity with the increase in PEO concentration
\item Solution Stirring: 24 $h$ of free diffusion followed by 96 $h$ of stirring at 30 $r p m$
\item 3 $m L$ syringe
\item 27 gauge type 304 stainless steel needle
\item Solution deposition rate: lower than 1 $\mu L / h$
\item needle-to-collector distance: 1 $m m$
\item Collector substrate: Pyrolyzed SU-8 carbon and Si
\item NFES process initiated by an air interference with a glass microprobe tip (1 to 3 $\mu m$ tip diameter) to overcome the surface tension
\item Time to produce a stable continuous jet: 45 $min$
\item Polymer jet initiated at 400-600 $V$ and dispensed at 200-400 $V$
\item Collector linear speed: 10-40 $m m / s$
\item The voltage turned on when the solution formed a full-sized droplet of 500 $\mu m$ diameter at the needle tip.
\end{itemize} &
\begin{itemize}[leftmargin=*]
\item Diameter: 50-425 $n m$
\end{itemize} &
\cite{Bisht2011}   \\ %Controlled Continuous Patterning of Polymeric Nanofibers on Three-Dimensional Substrates Using Low-Voltage Near-Field Electrospinning
\hline
\label{tbl:FloresCompare}
\end{tabular}
%\end{center}
\end{table}

\begin{table}[th]
\caption{Electrospun Polymer Solutions - Solution and Process Parameters}
%\begin{center}
% total table width eq 0.91\textheight = 100%
\begin{tabular}{
>{\raggedright\arraybackslash}p{0.130\textheight}
>{\raggedright\arraybackslash}p{0.130\textheight}
>{\raggedright\arraybackslash}p{0.090\textheight}
>{\raggedright\arraybackslash}p{0.350\textheight}
>{\raggedright\arraybackslash}p{0.180\textheight}
>{\raggedright\arraybackslash}p{0.020\textheight} }  
\hline
Polymer(s) & Solvent(s) & NFES Variant & Polymer Solution and Process Properties & Fiber Characterization & Ref. \\
\hline
Poly[2-methoxy-5-(2-ethylhexyloxy)-1,4-phenylenevinylene] (MEH-PPV) with Poly(ethylene oxide) (PEO) &
acetonitrile / toluene mixture (65 / 35); acetic acid / toluene (17 / 83); pure toluene &
Mechano electrospinning &
\begin{itemize}[leftmargin=*]
\item Concentrations:
    \begin{itemize}[leftmargin=*]
    \item MEH-PPV solution: 10 $m g$ of MEH-PPV in 2 $m L$ of toluene
    \item 500 $\mu L$ of MEH-PPV solution with 250 $m g$ of PEO in 3.5 $m L$ of acetonitrile / toluene (65 / 35)
    \item 500 $\mu L$ of MEH-PPV solution with 250 $m g$ of PEO in 3 $m L$ of acetic acid / toluene (17 / 83)
    \item The resulting MEH-PPV/PEO concentration is 1:100
    \end{itemize}
\item Solution Stirring: MEH-PPV solution stirred for 4 $h$; PEO solution stirred for 8 $h$; MEH-PPV/PEO solution stirred and ultrasonically agitated
\item Collector substrate: SiO2/Si (oxide thickness = 800 $n m$)
\item needle-to-collector distance: 500 $\mu m$
\item $\mu m$-diameter tip Tungsten spinneret in a 26 gauge needle
\item Solution deposition rate: 50 $\mu L / h$
\item Electrostatic voltage: around 1.3 $k V$
\item x-y stage velocity: 50 $c m / s$
\end{itemize} &
\begin{itemize}[leftmargin=*]
\item Distance between adjacent fibers: around 100 $\mu m$
\item Fiber diameter: around 100 $n m$
\end{itemize} &
\cite{Camillo2013} \\ %Near-field electrospinning of conjugated polymer light-emitting nanofibers
\hline
\label{tbl:FloresCompare}
\end{tabular}
%\end{center}
\end{table}

\begin{table}[th]
\caption{Electrospun Polymer Solutions - Solution and Process Parameters}
%\begin{center}
% total table width eq 0.91\textheight = 100%
\begin{tabular}{
>{\raggedright\arraybackslash}p{0.130\textheight}
>{\raggedright\arraybackslash}p{0.130\textheight}
>{\raggedright\arraybackslash}p{0.090\textheight}
>{\raggedright\arraybackslash}p{0.350\textheight}
>{\raggedright\arraybackslash}p{0.180\textheight}
>{\raggedright\arraybackslash}p{0.020\textheight} }  
\hline
Polymer(s) & Solvent(s) & NFES Variant & Polymer Solution and Process Properties & Fiber Characterization & Ref. \\
\hline
Poly(ethylene oxide) (PEO) &
Water &
Mechano electrospinning &
\begin{itemize}[leftmargin=*]
\item 7 wt \% PEO aqueous solution
\item Under room temperature at 1 $atm$
\item needle-to-collector distance: 500 $\mu m$
\item needle diameter: outer: 200 $\mu m$; inner: 100 $\mu m$
\item applied voltage for jet initiation: 1.5 $k V$
\item applied voltage for fiber deposition: 600 $V$
\item Mechanical drawing is applied by using a tungsten probe with 1 $\mu m$ tip diameter to poke inside the meniscus.
\item The probe is then rapidly pulled away from the polymer droplet to activate the continuous electrospinning process
\item polymer jet diameter: 3 $\mu m$
\item polymer feed rate: 0.1 $\mu L / h$
\item x-y stage velocity: 120 $m m / s$
\end{itemize} &
\begin{itemize}[leftmargin=*]
\item 108 $m$ yield in 15 $min$ with a fiber diameter of 709 $\pm$ 131 $n m$
\item Fiber diameter: around 49-74 $n m$ when applied voltage is 800 $V$
\end{itemize} &
\cite{Chang2008}   \\ %Continuous near-field electrospinning for large area deposition of orderly nanofiber patterns
\hline
\label{tbl:FloresCompare}
\end{tabular}
%\end{center}
\end{table}

\begin{table}[th]
\caption{Electrospun Polymer Solutions - Solution and Process Parameters}
%\begin{center}
% total table width eq 0.91\textheight = 100%
\begin{tabular}{
>{\raggedright\arraybackslash}p{0.130\textheight}
>{\raggedright\arraybackslash}p{0.130\textheight}
>{\raggedright\arraybackslash}p{0.090\textheight}
>{\raggedright\arraybackslash}p{0.350\textheight}
>{\raggedright\arraybackslash}p{0.180\textheight}
>{\raggedright\arraybackslash}p{0.020\textheight} }  
\hline
Polymer(s) & Solvent(s) & NFES Variant & Polymer Solution and Process Properties & Fiber Characterization & Ref. \\
\hline
Poly($\varepsilon$-Caprolactone) (PCL) &
\emph{Not applicable.} &
Melt Electrospinning Writing (MEW) &
\begin{itemize}[leftmargin=*]
\item Collector substrate: NCO-sP(EO-stat-PO)-coated glass slide surfaces
\item fibre diameters are between 5 and 30 $\mu m$
\end{itemize} &
\begin{itemize}[leftmargin=*]
\item 1
\item 2
\end{itemize} &
\cite{Dalton2015}  \\ % Additive manufacturing of scaffolds with sub-micron filaments via melt electrospinning writing Related content Patterned melt electrospun substrates for tissue engineering
\hline
\label{tbl:FloresCompare}
\end{tabular}
%\end{center}
\end{table}

%%%%%%%%%%%%%%%%%%%%%%%%%%%%%%%%%%%%%%%%%%%%%%%%%%%%%%%%%%%%%%%%%%%%%%%%%
\begin{table}[th]
\caption{Electrospun Polymer Solutions - Solution and Process Parameters}
%\begin{center}
% total table width eq 0.91\textheight = 100%
\begin{tabular}{
>{\raggedright\arraybackslash}p{0.130\textheight}
>{\raggedright\arraybackslash}p{0.130\textheight}
>{\raggedright\arraybackslash}p{0.090\textheight}
>{\raggedright\arraybackslash}p{0.350\textheight}
>{\raggedright\arraybackslash}p{0.180\textheight}
>{\raggedright\arraybackslash}p{0.020\textheight} }  
\hline
Polymer(s) & Solvent(s) & NFES Variant & Polymer Solution and Process Properties & Fiber Characterization & Ref. \\
\hline
% &  &  &  &  & \cite{Bisht2011}   \\ %Controlled Continuous Patterning of Polymeric Nanofibers on Three-Dimensional Substrates Using Low-Voltage Near-Field Electrospinning
% &  &  &  &  & \cite{Camillo2013} \\ %Near-field electrospinning of conjugated polymer light-emitting nanofibers
% &  &  &  &  & \cite{Chang2008}   \\ %Continuous near-field electrospinning for large area deposition of orderly nanofiber patterns
% &  &  &  &  & \cite{Dalton2015}  \\ % Additive manufacturing of scaffolds with sub-micron filaments via melt electrospinning writing Related content Patterned melt electrospun substrates for tissue engineering
 &  &  &  &  & \cite{Duan2017}    \\ %Helix Electrohydrodynamic Printing of Highly Aligned Serpentine Micro/Nanofibers.
 &  &  &  &  & \cite{Gupta2007}   \\ %Novel Electrohydrodynamic Printing of Nanocomposite Biopolymer Scaffolds
 &  &  &  &  & \cite{Huang2015}   \\ %Versatile, kinetically controlled, high precision electrohydrodynamic writing of micro/nanofibers
 &  &  &  &  & \cite{Jiang2018}   \\ %Electrohydrodynamic Direct-Writing Micropatterns with Assisted Airflow
 &  &  &  &  & \cite{Kim2018}     \\ %Characterization of 3D electrospinning on inkjet printed conductive pattern on paper
 &  &  &  &  & \cite{Lee2012}     \\ %Fabrication of Patterned Nanofibrous Mats Using Direct-Write Electrospinning
 &  &  &  &  & \cite{Liu2014}     \\ %Direct-write PVDF nonwoven fiber fabric energy harvesters via the hollow cylindrical near-field electrospinning process
 &  &  &  &  & \cite{Min2013}     \\ %Large-scale organic nanowire lithography and electronics
 &  &  &  &  & \cite{Pan2014}     \\ %Poly($\gamma$-benzyl $\alpha$, l-glutamate) in Cylindrical Near-Field Electrospinning Fabrication and Analysis of Piezoelectric Fibers
 &  &  &  &  & \cite{Pan2015}     \\ %Near-field electrospinning enhances the energy harvesting of hollow PVDF piezoelectric fibers
 &  &  &  &  & \cite{Song2015}    \\ %Patterned polydiacetylene-embedded polystyrene nanofibers based on electrohydrodynamic jet printing
 &  &  &  &  & \cite{Sun2006}     \\ %Near-Field Electrospinning
 &  &  &  &  & \cite{Wang2015}    \\ %Research on Multinozzle Near-Field Electrospinning Patterned Deposition
 &  &  &  &  & \cite{Wang2017}    \\ %Controllable deposition distance of aligned pattern via dual-nozzle near-field electrospinning
 &  &  &  &  & \cite{Wang2018}    \\ %Fabrication and evaluation of controllable deposition distance for aligned pattern by multi-nozzle near-field electrospinning
 &  &  &  &  & \cite{Xu2014}      \\ %Accuracy Improvement of Nano-fiber Deposition by Near-Field Electrospinning
 &  &  &  &  & \cite{Xue2014}     \\ %Rapid Patterning of 1-D Collagenous Topography as an ECM Protein Fibril Platform for Image Cytometry
 &  &  &  &  & \cite{Zheng2010}   \\ %Precision deposition of a nanofibre by near-field electrospinning
 &  &  &  &  & \cite{Zheng2014}   \\ %Electrohydrodynamic Direct-Write Orderly Micro/Nanofibrous Structure on Flexible Insulating Substrate
 &  &  &  &  & \cite{Zheng2012}   \\ %Polymer nanofibers prepared by low-voltage near-field electrospinning
\hline
\label{tbl:FloresCompare}
\end{tabular}
%\end{center}
\end{table}

%\addtolength{\oddsidemargin}{0.0in}
%\addtolength{\evensidemargin}{0.0in}
\addtolength{\topmargin}{0.0in}

\end{landscape}
\pagebreak[4]\global\pdfpageattr\expandafter{\the\pdfpageattr/Rotate 0}

%----------------------------------------------------------------------------------------
%	SECTION 1
%----------------------------------------------------------------------------------------



%-----------------------------------
%	SUBSECTION 1
%-----------------------------------


