% Chapter Template

\chapter{Problem Definition and Motivation} % Main chapter title

\label{Chapter:ProblemDefinitionandMotivation}

\subsubsection*{\color{mygray}[Chapter ready for review]}
% Describe the current situation of this problem.
% identifying the obstacles and the possible applications of your research.
% background of the problem (e.g. origins, previous attempts, etc.). 
% State the problem in detail mentioning all relative aspects, variables, relationships,
% stressing on the importance of finding a solution to the problem.

Carbon nanowires have been fabricated with a photoresist by multiple-photon polymerization techniques. However little is known about polymers that can produce conductive carbon nano-wires after pyrolysis. The problem relays on the fact that it had been known that most polymers are non-graphitic through pyrolysis \cite{Franklin1951}. In the past years photon polymerization processes have been applied to the fabrication of nano-structures with the use of a photoresist. \cite{Boer2014} Photon polymerization techniques deliver patterning resolutions with nano-scale tolerances for the production of highly detailed structures, \cite{Hribar2014} However, it is typical of photon polymerization processes to yield small objects with waste resins that are often toxic. \cite{Ovsianikov2012}

On the other hand, electrospinning has been acknowledged for as a process with promising results at nano-structure fabrication \cite{Boer2014}, yet there is little research regarding the implementation of electrospinning for the fabrication of carbon nano-wires. Electrospinning has the potential to be a more straightforward process for the design and fabrication of nano-structures, as it can achieve mass scale manufacturing in a continuous, simple and reproducible manner. \cite{Cardenas2017} shows that electrospinning can be implemented with ease for carbon nano-wire synthesis. Mechano-electrospinning, a new variant of electrospinning shows promising results in the production of ordered carbon nano-wires. As stated in \cite{Cardenas2017}, mechano-electrispinning is an early technology invention, and brings new challenges, such as the reproducibility of carbon nano-wire production.

Since electrospinning seems to be a better alternative for carbon nano-wire synthesis processes; and for that purpose of its implementation, it is required to develop polymer solutions that can be mechano-electrospun, photopolymerized and pyrolyzed into conducting carbon nano-wires. Carbon nano-materials have been subjected to research due to their various potential applications in diverse areas that take advantage of the nano-scale properties. \cite{Siddiqui2019} Carbon nano-materials are suitable for the catalysis, adsorption, carbon capture, energy and hydrogen storage, drug delivery, bio-sensing and cancer detection. \cite{Siddiqui2019} However most application are not currently feasible due to the lack of a continuous, simple and reproducible fabrication method. With the newly designed polymer solution, it would be possible to produce carbon nano-wires in large quantities, and therefore more applications will become feasible.


%----------------------------------------------------------------------------------------
%	SECTION 1
%----------------------------------------------------------------------------------------



%-----------------------------------
%	SUBSECTION 1
%-----------------------------------

