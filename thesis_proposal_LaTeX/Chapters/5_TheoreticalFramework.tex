% Chapter Template

\chapter{Theoretical Framework} % Main chapter title

\label{Chapter:TheoreticalFramework}

\subsubsection*{\color{mygray}[Chapter under work]}
% Background and state-of-the-art
% This involves a thorough state-ofthe-art bibliographic review, the building of a formal foundation, the study of related techniques, analysis and awareness of previous works and study of fundamental aspects that can help carry on your research.

\section{Photoresists}
The electronic industry requires sustainable raw material supply for its development \cite{Sutikno2016}. Photoresists are a type of raw material used in microelectronics, which is composed by four main elements: a polymer (resin), a photoactive compound, a solvent, and an additive \cite{Schuster2009}. The additive requires to be with a low molecular weight as it is intended to act as a photosensitive material. Photoresists are used within the manufacturing process of printed circuit boards \cite{Staab2011}. Photoresists are classified into two categories. The resist is defined as positive if the radiation exposed material is soluble in photoresist developer; otherwise, for negative photoresist the exposed material remains to stay in the photoresist surface as it crosslinks upon exposure \cite{Landis2011,Sharma2012}. In the manufacturing process of a semiconductor, the radiation sources which are often used in a lithography process are ultraviolet (UV) and X-ray \cite{Mekaru2015}.

The polymeric material is available on the broad market either in liguid or solid state; \emph{MicroChem Corp.} (Westborough, MA, USA) is the principal provider of SU-8 photoresist. SU-8 and similar photoresists are inexpensive with good adhesion on the semiconductor surface and high sensitivity \cite{Staab2011}. Epoxy resins are copolymer-thermosetting plastics which are normally produced by a chemical reaction process that involves epichlorohydrin and bisphenol-A compound \cite{Singla2010}. A epoxy-based polymer is typically used to produce patterns by lithography with the application of UV radiation. Lithography is a technique to transfer patterns from a mask and then transferred onto the substrate \cite{Landis2011,Xu2014}. SU-8 is a epoxy-based negative photoresist with the advantages of being inexpensive with good mechanical properties, good chemical resistance, and good electrical isolation \cite{Xu2014}. SU-8 photoresists are used in the production processes of MEMS \cite{Zhang2001}. Photoresist-wise, the contrast and quality level of UV radiation lithography is affected by the wavelengths of radiation sources. The higher the sensitivity of the material, the better is the lithography process as it absorbs radiation energy with ease to perform photochemical reactions in forming patterns \cite{Zhang2001}.

In summary, a photoresist is a "epoxy-based resin (polymeric) material which changes its dissolution rate in a liquid solvent, called a developer, under high energy radiation.`` \cite{Landis2011}

\section{Electro-Mechanical Spinning}
A number of techniques have being develop for the fabrication of nano-fibers, such as arc discharge [22], chemical vapor deposition, laser ablation [23], and vapor growth [21]. Nonetheless, those processes are expensive due to either the low product yield or the expensive equipment required. The electrospinning method can produce fibers with a range of diameters between 10 $n m$ and 10 $\mu m$ 

Diverse polymer patterning techniques have been developed to integrate and synthesize carbon nano-wires. A typical technique is electrospinning. Electrospinning requires the application of an electrostatic force to a polymer solution to spin fibers. The applied electric field, the solution conductivity, jet length, solution viscosity surrounding gas, flow rate and the collector geometry are important parameters that influence the fiber formation during the spinning. \cite{Nataraj2012} Two sub-techniques can be derived from electrospinning depending on the distance between the dispensing electrode and the collector. The process in which the electrospun jet can be controlled near the
tip is called NFES or near-field electrospinning. \cite{Cisquella-Serra2019} Moreover, if the distance between the collector and the dispensing needle is farther, the configuration is known as FFES or far-field electrospinning. \cite{Nataraj2012}

\section{Carbon nano-fibers}
Carbon nano-wires (CMWs) are known as long, thin strings with diameters between 10 and 1 thousand nm; composed mostly by carbon atoms aligned parallel to the long axis of the fiber. \cite{Nataraj2012} Carbon nano-wires are different from carbon nano-tubes, as CMWs are not composed by graphene sheets in cylindrical form. \cite{Nataraj2012}

%----------------------------------------------------------------------------------------
%	SECTION 1
%----------------------------------------------------------------------------------------



%-----------------------------------
%	SUBSECTION 1
%-----------------------------------


