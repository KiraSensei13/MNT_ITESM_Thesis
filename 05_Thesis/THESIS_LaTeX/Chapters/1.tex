% Chapter Template

\chapter{Introduction} % Main chapter title

\label{Chapter:1}

Carbon nano-materials are subjected of great interest for research purposes due to their various potential applications in diverse areas that take advantage of the nano-scale properties. Carbon nano-materials are suitable for catalysis, adsorption, carbon capture, energy and hydrogen storage, drug delivery, bio-sensing, and cancer detection. \cite{Kenry2017, BaudritJ2017} Some matchless properties that allow carbon nano-materials to be utilized within multiple functionalities include high porosity, distinguished structures, uniform morphologies, high stability, high magnetic properties, and high conductivity. \cite{McCreery2008, Geim2011, Zhu2010, Katsnelson2008, Li2008, Geim2007, Geim2009, Siddiqui2019}

This document bestows a thesis project to perform research to engineer a polymer solution to contribute towards the long-term goal of achieving mass scale manufacturing of high conductive glass-like carbon nano-wires with a control of the jet diameter in an inexpensive, continuous, simple and reproducible manner. This thesis discusses several manufacturing processes such as near field electrospinning, photo-polymerization, pyrolization, and carbonization, as they have shown to be promising methods for the fabrication of carbon nano-materials. \cite{Cardenas2017} See Figure \ref{fig:fabricationFlowChart}. A number of processes have been developed for specific purposes of polymeric nano-fibres, some include surface deposition, composites, and chemical adjustments. Polymeric nano-fibers must be also pyrolyzed to generate carbon nano-wires with conductive capabilities \cite{Madou2011} for electrochemical sensing and energy storage purposes.

\begin{figure}[!th]
\centering
\includegraphics[width=1.0\textwidth]{./Figures/FabricationProcess.png}
\decoRule
\caption[Fabrication Process of Carbon Nano-wires]{Fabrication process and characterization techniques of conductive carbon nano-wires to achieve through the dissertation.}
\label{fig:fabricationFlowChart}
\end{figure}

Nanotechnology has led to the study of different polymer patterning techniques to integrate carbon nano-wires structures. One technique is known as far-field electrospinning (FFES), a process in which electrified jets of polymer solution are dispensed to synthesize nano-fibres which can then be pyrolyzed. One sub-technique derived from electrospinning is near-field electromechanical spinning or NFEMS. Unlike FFES, NFEMS has proved to deliver sufficient spatial control for patterning polymeric nano-fibres. \cite{Cardenas2017}

The present work was proposed to continue the work done by others \cite{Cardenas2017, Flores2017} in regards to the synthesis of carbon nano-wires. Previous work includes the fabrication of suspended carbon nano-wires by two methods: electro-mechanical spinning and two-photon polymerization with a photoresist. \cite{Cardenas2017, Flores2017} This work is intended to focus on electro-mechanical spinning processes only, to bring off polymer solutions that can be electrospun by NFEMS to yield polymer fibers than can hopefully be crosslinked by UV light and the pyrolyzed and pyrolyzed into conducting carbon nano-wires. The polymer solutions described by Cárdenas and Flores \cite{Flores2017, Cardenas2017} were used as benchmarks and starting points for the present studies.

Traditional near-field electrospinning or NFES allows large-scale manufacturing combined with spatial control of material deposition. \cite{Madou2011} However, the reported efforts required the use of electric fields in excess of 200 kV/m for continuous operation, this seems to contradict the previous sentence. \cite{Madou2011} concluded that the state-of-the-art fabrication processes for polymer nano-fibers are still lacking in terms of precision, cost, speed and throughput.

%----------------------------------------------------------------------------------------
%	SECTION 1
%----------------------------------------------------------------------------------------
\section{Carbon Nanowires Research Developments in Terms of Published Papers, Synthesis and Fabrication}

% \cite{Rauti2019}

Nanotechnology's ability to control and piece together materials at the nano-scale has enabled the development of various carbon nano-materials and carbon nano-structures, such as nano-dots, nano-fibres, nano-tubes and nano-wires. \cite{Posthuma-Trumpie2012, Zhang2009, DeVolder2011, Cao2011} This section focuses on the applications at the micro-scale and nano-scale levels, as well as the current research of carbon-based nano-materials (CBNs).

\subsection{Carbon and carbon-based nanomaterials}

\begin{figure}[!th]
\centering
\includegraphics[scale=0.37]{./Figures/CBNfingerprint.png}
\decoRule
\caption[Fingerprint of Carbon-based Nano-materials]{Molecular to meso-scale structural features of synthetic polymers influence the emergence of specific micro-structural features in polymer-derived carbon materials after pyrolysis.}
\label{fig:CBNfingerprint}
\end{figure}

Carbon is a versatile element capable of forming a number of bonds with other elements or with itself. Cabon-based nano-materials (CBNs) exist in diverse forms, depending on the precise values of each degree of freedom that specify the material properties at multiple scales. Hybridization, crystallization, percolation, anisotropy, porosity, impurities and imperfections are some of the relevant features that determine the CBN set of properties. The combination of these features at the micro- and meso-scale burst a variety of macro-scale properties that comprise the CBN fingerprint (\ref{fig:CBNfingerprint}). The interminable collection of possible CBN fingerprints range from soft, conductive lubricants to very hard, low conductivity solids \cite{Hugh1994}; and from black colour, bulks to transparent, disordered thin films. \cite{McCreery2008} Figures \ref{fig:carbonAllotropes} and \ref{fig:carbonAllotropesDiagram} show the existence of different allotropes based on carbon orbitals which have the ability to hybridize in sp1, sp2 and sp3 configurations, assembling different carbon allotropes.

\begin{figure}[!th]
\centering
\includegraphics[scale=0.30]{./Figures/carbonAllotropes.png}
\decoRule
\caption[Carbon sp-hybrid Nano-materials]{Three carbon allotropes (diamond, carbyne and graphene) are the building blocks of additional deriving carbon-based materials such as fullerenes, porous carbon and glass-like carbon.}
\label{fig:carbonAllotropes}
\end{figure}

\begin{figure}[!th]
\centering
\includegraphics[scale=0.65]{./Figures/carbonAllotropesHybridization.png}
\decoRule
\caption[Ternary Diagram of Carbon Allotropes based on sp Content]{Ternary phase diagram of amorphous carbon regions based on hybridization degree. Adapted from \cite{Heersche2007, Heimann1997, Belenkov2003, Fedel2013, Razeghi2019, AlstrupJensen2015, Vajtai2013}.}
\label{fig:carbonAllotropesDiagram}
\end{figure}

In terms of porosity, CBNs exhibit different properties according to the degree of 'open' and 'closed' pores. A 'closed pore' is a void or empty space in solid materials where a discontinuity is present within the array of atoms and molecules. On the other hand, an 'open pore' refers to a void which is connected to the outer surface of the solid, in other words a 'open pore' is a 'closed pore' with an opening to the external surface. \cite{Marsh1989} Figure \ref{fig:carbonAllotropesPorosityNOrder} shows a classification of carbon allotropes according to their porosity.

\begin{figure}[!th]
\centering
\includegraphics[scale=0.65]{./Figures/carbonAllotropesPorosityNOrder.png}
\decoRule
\caption[Ternary Diagram of Carbon Allotropes based on Porosity and Structural Order]{Ternary phase diagram of amorphous carbon regions based on structure order and porosity. Regions are colored by the degree of crystalline order within the carbon structure. White represents highly ordered structures, whereas white represents disordered structures. \cite{Marsh1989, Hugh1994}}
\label{fig:carbonAllotropesPorosityNOrder}
\end{figure}

Thermal conductivity and electrical conductivity decrease with increasing porosity due to the reduced amount of material to conduct electrons and energy. Furthermore, porosity negatively affects the mechanical properties like strength and elastic modulus as it reduces the volume in which stresses are distributed. \cite{Hugh1994} Moreover, stresses are concentrated at the pores which makes the material prone to mechanical failure. \cite{Marsh1989, Hugh1994}

Due to the versatility and variety of CBNs, CBNs have been fabricated and implemented for various purposes. \cite{Geim2011, Katsnelson2008, Li2008, Geim2007, Geim2009, Siddiqui2019}. For instance, field effect transistors (FET) have been studied by Novoselov \cite{Novoselov2004} and Heersche et. al. \cite{Heersche2007}. Carbon FET devices have reported field-effect mobility one order of magnitude higher than that of silicon FETs. Other literature suggests CBNs to be favorable to detect a variety of gases and bio-molecules. \cite{Schedin2007, Ohno2009} As molecules are absorbed by the CBN, the carrier density and electrical resistivity of the carbon material changes. Moreover, CBNs have showed good performance in applications in energy (prevent wastage of energy), water (purification) and diagnostics (lab-on-chip systems and nano-sensors). \cite{Cao2011, Khanna2016} As mentioned above, the morphology of CBNs has an impact on the electrochemical and mechanical properties. \cite{Marsh1989, Hugh1994, Guo2018} In this regard, carbon nano-structures, such as nano-wires \cite{Kundu2019, Bencheikh2019, Bencheikh2019}, have been fabricated to achieve improved electrochemical characteristics.

\subsection{Carbon Nano-wires}

As depicted in Figure \ref{fig:carbonAllotropesDiagram}, carbon nano-wires (CNFs) have been classified as linear, sp2-rich structures. \cite{Heersche2007, Heimann1997, Belenkov2003, Fedel2013, Razeghi2019, AlstrupJensen2015, Vajtai2013} Nano-fibers own good electrical, optical and mechanical characteristics, however those properties are highly dependent on the morphology of the fibers. \cite{Dresselhaus2007} The material properties of 1D nano-structures depend on fiber diameter, porosity, crystallinity degree and crystallite orientation. Consequently, the fabrication parameters and environment conditions have an impact on the reproducibility of high quality fibers. \cite{Dresselhaus2007} Carbon nano-fibers (CNFs) have diameters of several micrometers (Figure \ref{fig:diameterComparisonOfCarbonFibers}) and are different from carbon nano-tubes (CNT). \cite{Weil1992, Huang2009, chung2012carbon, Subramoney1997, Dresselhaus2000} Unlike carbon nano-tubes with hollow cores, carbon nano-fibers can be represented as stacked layers along the thread length. \cite{Dresselhaus2000, Rodriguez1993, Yoon2004} The stacked geometry of carbon nano-fibers results in unique electrical, chemical and mechanical properties. \cite{Liu2009, Endo2003, Yokozeki2009} Unlike CNFs, carbon nano-tubes inherent problems such as high cost and low effective surface area, which limit their practical use. \cite{Vajtai2013}

\begin{figure}[!th]
\centering
\includegraphics[scale=0.65]{./Figures/carbonFiberDiameterComparison.png}
\decoRule
\caption[Diameter comparison of various types of fibrous carbon materials]{Various types of fibrous carbon materials bear different characteristics according to their molecular structure. Adapted from \cite{Vajtai2013}}
\label{fig:diameterComparisonOfCarbonFibers}
\end{figure}

Carbon nano-wires have been used for the improvement of power density and specific energy in lithium-ion batteries. \cite{Frackowiak2002, Endo2000, Winter1998} Authors posit that the performance and capacity of Li-ion batteries depend on the CNF structure and texture. Through the right combination of electrospinning and carbonization parameters, electrically conductive, mechanically tough and low diameter fibers have been achieved by Yoon et al. \cite{Yoon2004}. Yoon reported 431 mili-ampere-hour per gram batteries with vitreous carbon nanofibers. Yoon states that the battery capacity highly depends on the pyrolysis process parameters as the morphology of the fiber develops pores and hence different surface properties. CNFs supercapacitors have been investigated as energy storage devices due to their high power bearability and long lifecycles. \cite{Endo2001, Frackowiak2001, Pandolfo2006, Conway1999} The studies' authors posit that carbon nano-fibers can be implemented as high-power supercapacitors due to their large surface area and high electrical conductivity.

On the other hand, the low reactivity and unique morphology of CNFs make them promising catalyst supports for metal nano-particles. \cite{Choi2002, Li2002, Planeix1994} It is well known that the morphology and nano-structure of the supporting material are the main factors that prevent agglomeration of nano-particles. \cite{RomanMartinez1995, Serp2008} Moreover, in bone tissue scaffold applications, collagen is the most popular scaffold. However, collagen scaffolds bring xenogenicity issues which leads to disease transfer or immunogenic reactions, besides its inhability to preserve its shape once placed in the body. \cite{Bach1998, Butler1998, Delustro1990, Chachques2008, Atala2006, Glowacki2008, Valarmathi2008, Faraj2007} Currently, carbon fibers have been studied for bone tissue scaffold, however early attempts yield too thick fibers for cell cultivation and tissue regeneration. \cite{Visuri1991, Parsons1989} As depicted in previous research of CNFs for different applications, fiber morphology seems to have a significant impact on their performance. 

Typically, carbon nano-fibers (CNFs) are synthesized by a combination of a patterning process and a pyrolysis process. Electrospun CNFs have characteristics such as high surface area, thin morphology with nano-scale diameters. The properties of electrospun fibers allow CNFs to be implemented in nano-sensing devices, energy storage applications, and tissue scafflods. \cite{Khanna2016, Ramakrishna2005, Reneker2000, Norris2000, Vozzi2002, Kim2003, Dersch2005} Several patterning techniques have been attempted to achieve the desired fiber morphology. In addition to electrospinning, CNFs have been also fabricated by two-photon polymerization (TPP) and photo-lithography techniques. \cite{CardenasBenitez2019} Cardenas et al. implemented TPP and conventional UV lithography to study the fabrication of CNFs within carbon micro-electromechanical systems (C-MEMS). The fabrication of these kind of carbon devices has been previously reported for techniques, such as electrospinning and photoresist patterning by photolithography using SU-8. The typical fabrication process of C-MEMS begins with a spin-coating of a photoresist unto a substrate (typically SU-8), followed by patterning techniques with UV-exposure by photolithography. Followed by the development of the desired features. Finally, the device is carbonized in a pyrolysis furnace in an inert environment. \cite{Pramanick2018a}

Near-field electrospinning can be regarded as a complementary technique, by which polymeric nanofibers can be produced, since the structural geometries created by photolithography are restricted by the diffraction limit. \cite{Pramanick2018a, Okazaki1991} SU-8 is designed to produce vitrous carbon structure via photolithography, it is not design for electrospinning procedures as it lacks the right viscosity and solution conductivity. Cardenas \cite{Cardenas2017} and Flores \cite{Flores2017} have adapted the SU-8 formulation by the addition of tetrabutylammonium tetrafluoroborate (TBF) and poly(ethylene oxide) (PEO). TBF was added to increase the solution conductivity and PEO provides the required viscosity. Both additives are required to yield smooth solution flow during electrospinning. Figure \ref{fig:su8Components} illustrates the ingredients that comprise the SU-8 formulation.

\begin{table}[!th]
\centering
\caption[Polymer Solutions from Previous Work]{Polymer Solutions from Previous Work \cite{Cardenas2017, Flores2017}}
\begin{tabular}{cccc}
\hline
\textbf{Sample} & \multicolumn{3}{c}{\textbf{Concentration} $wt\%$} \\
\hline
{} & \textbf{SU-8} & \textbf{PEO} & \textbf{TBF} \\
1  & 99.25         & 0.25         & 0.50         \\
2  & 99.00         & 0.50         & 0.50         \\
3  & 98.75         & 0.75         & 0.50         \\
4  & 98.50         & 1.00         & 0.50         \\
\hline
\end{tabular}
\label{tab:previousPreparedPolymerSolutions}
\end{table}

The thinnest fibers fabricated by Cardenas \cite{Cardenas2017} were achieved with sample 1 of Table \ref{tab:previousPreparedPolymerSolutions}, with the following characteristics: a) Fiber yield rate of $81\%$; b) Fiber diameter before pyrolysis of $4.966 \mu m$; c) Fiber diameter after pyrolysis of $204 nm$; d) Average fiber length of $60.5 ± 4.3 \mu m$; and e) Fiber electrical resistance from $407 K \Omega$ to $1.727 M \Omega$. Cardenas results have areas of opportunity regarding the fiber yield rate and the high variability on the fiber electrical resistance. These undesirable characteristics could be a consequence of the addition of PEO to the solution. SU-8 based vitreous carbon is obtained after a pyrolysis process in which the oxygen already present in the SU-8 formulation allows the formation of close pores during annealing. However, the further addition of oxygen content present in the PEO molecules may be the cause of the low yield rate and high variability in electric resistivity from sample to sample.

\begin{figure}[!th]
\centering
\includegraphics[scale=0.50]{./Figures/su8Components.png}
\decoRule
\caption[Components of SU-8 2000 Series Resists]{Components of SU-8 2000 Series Resists. Adapted from \cite{Microchem2012}}
\label{fig:su8Components}
\end{figure}

%----------------------------------------------------------------------------------------
%	SECTION 2
%----------------------------------------------------------------------------------------
\section{Problem definition and motivation}

%Qué metodos se han utilizado? (SU-8)
%Porqué se quiere fabricar las fibras de carbono
%Para qué aplicaciones se han usado
%Porqué es importante la conductividad
%Porqué es importante que sea puro carbono

The role of carbon nano-wires in nano-sensor devices play an important role, as portable instruments require light-weight and small-sized components. \cite{Khanna2016} Table \ref{tab:advantagesOfNanosensors} lists some advantages of nano-sensors that can be accomplished by the fabrication of CNFs via near-field electrospinning and a thermal treatment in an inert environment.

\begin{table}[!th]
\centering
\caption[Advantages of Nano-sensors]{Advantages of Nano-sensors. Adapted from \cite{Khanna2016}}
\begin{tabularx}{\textwidth}{lX}
\hline
\textbf{Advantage} & \textbf{Description} \\
\hline
High sensitivity & More accuracy, single molecule detection \\
Small size & Light-weight, portability, low-power consumption, small sample size, reduced sample preparation, and ease of use \\
Low response time & High-frequency, real time analysis \\
Low cost & Disposable devices \\
\hline
\end{tabularx}
\label{tab:advantagesOfNanosensors}
\end{table}

Sensors of small size require less time to output a stable signal as signals require less time to travel shorter lengths, hence signal noise is also reduced. Nano-sized sensors allow data collection and measurements to be performed in real time at faster speeds. \cite{Khanna2016} The nano-scale also allows sensors to increase the active surface area, enabling the absorption and detection of analytes at low concentrations. \cite{Khanna2016} Conventional sensors are bulky and require higher amounts of power to operate. In gas sensing, neither a large sensing surface or a large sample is required to get a readable output signal from the sensor. Power consumption can be saved by reducing the thermal mass of the sensor. \cite{Khanna2016} Furthermore, if several gases are to be detected, an array of several gas sensors are to be assembled into an array. A multi-gas sensor array can increase the size and cost, whereas an array of gas nano-sensors (each functionalized to detect a specific analyte) can be implemented into a single device. \cite{Khanna2016} Nano-sensors can be classified by the kind of energy or physical phenomena that is detected, as depicted in Table \ref{tab:classificationOfNanosensors} for instance: biological, mechanical, thermal, chemical, and optical sensors. \cite{Khanna2016, BaudritJ2017}

\begin{figure}[!th]
\centering
\includegraphics[scale=0.50]{./Figures/typesOfNanoSensors.png}
\decoRule
\caption[Types of Nano-sensors]{Diagram examples of carbon-based nano-sensors. Adapted from \cite{BaudritJ2017}}
\label{fig:diameterComparisonOfCarbonFibers}
\end{figure}

\begin{table}[!th]
\centering
\caption[Classification of Nano-sensors]{Classification of Nano-sensors. Adapted from \cite{Khanna2016}}
\begin{tabularx}{\textwidth}{lX}
\hline
\textbf{Classification} & \textbf{Phenomena / Energy} \\
\hline
Mechanical & Position, acceleration, stress, strain, force, pressure, mass, density, viscosity, moment, torque \\
Acoustic & Wave amplitude, phase, polarization, velocity \\
Optical & Absorbance, reflectance, fluorescence, luminescence, refractive index, light scattering \\
Thermal & Temperature, flux, thermal conductivity, specific heat \\
Electrical & Charge, current, potential, dielectric constant, conductivity \\
Magnetic & Magnetic field, flux, permeability \\
Chemical & Components (identities, concentrations, states) \\
Biological & Biomass (identities, concentrations, states) \\
\hline
\end{tabularx}
\label{tab:classificationOfNanosensors}
\end{table}

Carbon nanowires have been fabricated with a photoresist by two-photon polymerization techniques. However little is known about polymers that can produce conductive carbon nano-wires after pyrolysis, as it is generally believed that most polymers do not form significant amounts of graphitic carbon when carbonized.
%The lack of research relays on the fact that in the past years, it was assumed that most polymers are non-graphitic through pyrolysis \cite{Franklin1951}.
In the past, photopolymerization processes have been applied to the fabrication of nano-structures with the use of an epoxy based photoresist. \cite{Boer2014} Photopolymerization techniques deliver patterning resolutions with nano-scale tolerances through two-photon lithography for the production of highly detailed structures \cite{Hribar2014}.

On the other hand, electrospinning has been classified as a process with promising results at nano-structure fabrication \cite{Boer2014}, yet there is little research regarding the implementation of electrospinning for the fabrication of carbon nano-wires. Electrospinning has the potential to be a more straightforward process for the design and fabrication of nano-structures, as it can achieve mass scale manufacturing in a continuous, simple and reproducible manner. Cardenas \cite{Cardenas2017} showed that electrospinning can be implemented with ease for carbon nano-wire fabrication. Mechano-electrospinning, a new variant of electrospinning shows promising results in the production of ordered carbon nano-wires. As stated in \cite{Cardenas2017}, mechano-electrospinning is a recent technology invention and brings new challenges, such as the reproducibility of carbon nano-wire production. Furthermore, the study of a new fabrication process to produce carbon nano-wires that involves mechano-electrospinning will enable spatial control of the fiber deposition.

Since electrospinning seems to be a better alternative for carbon nano-wire fabrication processes; and for that purpose of its implementation, it is required to develop polymer solutions that can be mechano-electrospun, photopolymerized and pyrolyzed into conducting carbon nano-wires. Most applications of carbon-based materials are not currently feasible due to the lack of a continuous, simple and reproducible fabrication method with inexpensive processes. With the newly designed polymer solution, it would be possible to produce carbon nano-wires in large quantities, and therefore more applications will become feasible. On the other hand, the new technique will overcome some limitations of other methods such as lithography. For instance, patterns created by lithography processes cannot be originated, only replicated, all constituent points of the pattern can only be addressed at the same time, and the process requires the pattern to be encoded into a mask. \cite{Landis2011}

%----------------------------------------------------------------------------------------
%	SECTION 3
%----------------------------------------------------------------------------------------
\section{Hypothesis}

The viscoelastic properties of polymer solutions along with synthesis parameters
%(stage velocity, voltage, dispense rate)
can be controlled through rheological analyses to obtain low voltage electrospun-able, photopolymerizable and graphitizable solutions for the fabrication conductive of carbon nano-wires.
% with specified dimensions (diameter and length).
The viscoelastic properties of polymer solutions along with synthesis parameters can be modified by replacing the PEO (Poly(ethylene) oxide) component within the existing polymer solutions described in Flores \cite{Flores2017} and Cardenas \cite{Cardenas2017} work. PEO is to be replaced as its only purpose is to allow the electrospinning process to take place, but no benefit is obtained from it after pyrolysis. The hypothesis is that oxygen-less polymers will yield carbon nano-wires of better quality than those made from PEO blends, therefore the study is to verify the eletrospinnability of high carbon content oxygen-less polymers in solution.

%----------------------------------------------------------------------------------------
%	SECTION 4
%----------------------------------------------------------------------------------------
\section{Research Questions}

\begin{itemize}
	\item{
	Is there any evidence of conductive carbon nano-wire fabrication though electrospun-able and pyrozable polymer solutions?
	}
	\item{
	What are the process parameters to consider/control for the fabrication processes of carbon nano-wires? 
	}
	\item{
	What viscoelastic properties are to be controlled/tested to deliver an electrospun-able and pyrozable polymer solution?	
	}
	\item{
	What are the optimal fabrication parameters for the synthesis of carbon nano-wires through near-field electromechanical spinning?	
	}
	\item{
	What materials can be used to ease the electrospinning process and favor the carbon nano-wire properties after pyrolysis? 
	}
\end{itemize}

%----------------------------------------------------------------------------------------
%	SECTION 5
%----------------------------------------------------------------------------------------
\section{Objectives}

\subsection{General objective}
Formulate polymer solutions by selection of linear high-molecular weight polymers and solvents and then match their viscoelastic properties to those of the benchmark SU-8/PEO solution to select the polymer/solvent combinations that have the greatest protential to replace or modify the SU-8/PEO formulation for the fabrication of microscopic polymer fibers that may be converted to conductive suspended carbon nano-wires.

\subsection{Specific objectives}

\begin{itemize}
	\item{
	Propose polymer solutions that can be electrospun by Near-Field Electrospinning.
    }
    \item{
    Through rheological analyses, determine if polymer solutions can have comparative viscoelastic properties to those of the SU-8/PEO benchmark.
    }
    \item{
    Learn how the diameter of the electrospun polymer fiber can be controlled by appropriate tuning of the NFES parameters and solution properties. 
    }
    \item{
    Propose alternatives to the SU-8/PEO benchmark formulation for the production of microscopic polymer fibers with potential for the fabrication of carbon nano-wires. 
    }
\end{itemize}

%----------------------------------------------------------------------------------------
%	SECTION 6
%----------------------------------------------------------------------------------------
\section{Dissertation Outline}
The dissertation is organized as follows. Chapter 1, an introduction to carbon-based nanomaterials is presented. The applications and characteristics of carbon structures are listed with an emphasis on carbon nano-wires. Chapter 2 is comprised by a review of the electrospinning process. The process parameters such as process variables, ambient parameters and solution properties and their influence in fibers formation are studied. Data collection of near-field electrospinning publications was done to execute an adimensional analysis to describe and predict the fiber diameterfrom the process parameters. Chapter 3 focuses on the selection of candidate polymer-solvent combinations to replace the PEO-SU-8 formulation. Rheological tests (frequency sweeps) were done to study the visoelasticity of polymer solutions. Chapter 3 estimates the optimal polymer concentrations to fabricate continuous fibers through NFES. Chapter 4 presents the fabrication of polymeric fibers. The near-filed process parameters, materials and methods are discussed, where a replicate of experiments of PEO solutions from literature is used as an experimental control. The last chapter shows the results of the fiber characterization of different sets of polymer solutions, though an optical microscope. Finally, the conclusions of this work and the considerations to future works are presented.