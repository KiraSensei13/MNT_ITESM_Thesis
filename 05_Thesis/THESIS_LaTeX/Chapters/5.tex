% Chapter Template

\chapter{Concluding Remarks} % Main chapter title

\label{Chapter:5}

%----------------------------------------------------------------------------------------
%	SECTION 1
%----------------------------------------------------------------------------------------
\section{Conclusions}

As stated within the introductory chapter \ref{Chapter:1}, this thesis is to verify the near-field electrospinnability of high-carbon oxygen-less polymers in solution. The spinnability of the tested solutions is used as a screening to choose candidate formulations that can replace the PEO in SU-8 solution. A near-field electrospinning literature review was used to learn about the effect of the process parameters and solution properties in the control of the fiber diameter. The review analyses (data analysis, parameter correlation, and adimensional analysis) declare that the polymer concentration is the parameter with the biggest impact on the fiber diameter, followed by the applied voltage, the working distance, and the zero-shear viscosity.

Rheological analyses were performed to determine the viscoelastic properties of the candidate polymer solutions. Zero-shear viscosities of the solutions in interest are measured  to find solutions with spunable polymer concentrations. As explained in the following, viscosity and polymer concentration have a significant impact on the spinnablity of polymer solutions, but they are not sufficient by themselves. As demonstrated by the experiments to fabricate fibers through NFES, PSMS in DMF and PSB in NMP formulations were not spinnable at their critical concentrations.

A series of experiments were carried out in order to find a correlation between the rheological properties of different polymer solutions in near-field electro-mechanical spinning for carbon structures. Flow curve measurement tests where carried out in an oscillatory rheometer to estimate the critical concentrations of various polymer-solvent systems at which they are spinnable by NFES. Since the formulation is 0.25 $wt\%$ PEO in SU-8 2002 has been studied in the past and is known to yield good polymer solutions but fails at the pyrolysis, this work intends to find spinnable and pyrolyzable systems as well as a method to discover new spinnable polymer solutions.

From the flow curve measurements, the zero-shear viscosity was estimated using the Carreau-Yassuda model. For all solutions, a shear thinning behavious was noticed. It was found that the viscosity-concentration plot is a good method to find the critical concentration at which the solution is able to produce fibers through NFES. However the method failed at calculating the spinnable concentration of PSMS. An significant takeaway is that a polymer does not have elastic properties in the pellet form, it will probably not work as a NFES polymer precursor.

Fibers were fabricated at different applied voltages for each set of solutions in order to very that in NFES fiber diameter increases with increasing applied voltage, whereas the inverse relationship is true for FFES. Other parameters such as working distance, stage velocity, nozzle diameter, flow rate were kept constant for all experiments; however, for low applied voltages the jet shall be initialed by manually breaking the polymer drop. the thickest average fiber diameter was achieved with the PS in THF system ($\Phi_0 = 600 V$, $D_{fiber} = 5.304 \mu m$) and the thinnest was $0.976 \mu m$ using the PVK in CHL system and an applied voltage of $200 V$. It was proved that attainable rehological information can be used to modify the NFES process parameters to yield the desired fiber morphology.

Moreover a data analysis was done on the NFES publications of the last 13 years to identify relationships between fiber diameter and the process parameters. For instance, it was confirmed that thin fibers are achived with los polymer concentrations, small nozzle diameters, low applied voltage, slow flow rates and high stage velocities. Also, using the collected data and the input of Helgeson's work \cite{Helgeson2007}, an dimensionless analysis was done to predict fiber diameters with easy to get parameters.

Finally, the fabricated fibers can be classified into three groups: a) the control sample (PEO in SU-8); b) poor quality fibers (PS in THF, and PSB in THF and DMF); and c) good quality fibers (PVK in CHL and PVK in SU-8). PS and PSB fibers have rough textures and low adhesion to the collector substrate. The rough surface and low adhesion is due to the rapid solvent volatilization, which results in fibers to solidify before landing on the substrate. The fast solidification minimize the fiber adhesion to the collector, causing the fibers to slip, and therefore the mechanical drag force in the fibers is reduced. Due to the low influence of the mechanical force, fibers are not stretched by the moving stage, which may explain the rough surface.

On the other hand, the formulation regarding PVK deposited smooth fibers with good adhesion to the substrate, similar to the PEO/SU-8 solution. The thinnest fibers ($700 \textrm{ nm}$ in diameter) were achieved PVK in CHL formulation. Moreover, the PVK in SU-8 solution seems to be the formulation to replace the PEO in SU-8 solution. Assuming that the additional oxygen content in PEO negatively affects the fiber yield rate and electrical resistivity variance due to degassing during pyrolysis, the PCK does not contain additional oxygen content and has sp2-hybridized carbon which can promote the formation of graphitic carbon during pyrolysis. The already oxygen in SU-8 is necesary to create closed pores during annealing, characteristic of glass-like carbon.

\section{Future work}

Apart from the work done, this dissertation opens pending research to enable NFES for the fabrication of carbon structures. The following lists future work that could be done as a continuation of this thesis.

\begin{itemize}
\item Helgeson's model \cite{Helgeson2007} was thought to work with far-field electrospinning, hence the deviation of the NFES data from the model trend. For an accurate NFES fiber diameter prediction, the mechanical stresses introduced by the moving stage shall be considered in the model (Equation \ref{eqn:ohnesorgeNumberRelationship}). 
\item This work verifies the electro-spinnability of four new formulations, however fibers were not carbonized into carbon structures. Further work shall study the pyrolysis process of the proposed fibers to get carbon structures with good electrical conductivity. A photo-polymerization process could be introduced before pyrolyzation to increase the order of the molecules and achieve carbon with higher conductivity.
\item Near-field electrospinning solutions require specific viscosities to initialte a polymer jet. The viscosity-concentration plot is a helpful tool to estimate the critical spinnable concentration of a polymer-solvent system. However there is room for improvement as this method only considers reological data. Other methods could be adopted to better tune other process parameters such as stage velocity, and applied voltage.
\end{itemize}



% enhance conductivity with photopolymerization, CNT streching, ...
