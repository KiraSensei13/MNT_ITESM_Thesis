%%%%%%%%%%%%%%%%%%%%%%%%%%%%%%%%%%%%%%%
% Deedy - One Page Two Column Resume
% LaTeX Template
% Version 1.2 (16/9/2014)
%
% Original author:
% Debarghya Das (http://debarghyadas.com)
%
% Original repository:
% https://github.com/deedydas/Deedy-Resume
%
% IMPORTANT: THIS TEMPLATE NEEDS TO BE COMPILED WITH XeLaTeX
%
% This template uses several fonts not included with Windows/Linux by
% default. If you get compilation errors saying a font is missing, find the line
% on which the font is used and either change it to a font included with your
% operating system or comment the line out to use the default font.
% 
%%%%%%%%%%%%%%%%%%%%%%%%%%%%%%%%%%%%%%
% 
% TODO:
% 1. Integrate biber/bibtex for article citation under publications.
% 2. Figure out a smoother way for the document to flow onto the next page.
% 3. Add styling information for a "Projects/Hacks" section.
% 4. Add location/address information
% 5. Merge OpenFont and MacFonts as a single sty with options.
% 
%%%%%%%%%%%%%%%%%%%%%%%%%%%%%%%%%%%%%%
%
% CHANGELOG:
% v1.1:
% 1. Fixed several compilation bugs with \renewcommand
% 2. Got Open-source fonts (Windows/Linux support)
% 3. Added Last Updated
% 4. Move Title styling into .sty
% 5. Commented .sty file.
%
%%%%%%%%%%%%%%%%%%%%%%%%%%%%%%%%%%%%%%%
%
% Known Issues:
% 1. Overflows onto second page if any column's contents are more than the
% vertical limit
% 2. Hacky space on the first bullet point on the second column.
%
%%%%%%%%%%%%%%%%%%%%%%%%%%%%%%%%%%%%%%


\documentclass[]{deedy-resume-openfont}
\usepackage{fancyhdr}
 
\pagestyle{fancy}
\fancyhf{}
 
\begin{document}

\vfill

%%%%%%%%%%%%%%%%%%%%%%%%%%%%%%%%%%%%%%
%
%     LAST UPDATED DATE
%
%%%%%%%%%%%%%%%%%%%%%%%%%%%%%%%%%%%%%%
\lastupdated

%%%%%%%%%%%%%%%%%%%%%%%%%%%%%%%%%%%%%%
%
%     TITLE NAME
%
%%%%%%%%%%%%%%%%%%%%%%%%%%%%%%%%%%%%%%
\namesection{Osamu}{Katagiri}{ \urlstyle{same}\href{https://www.katagiri-mx.com/}{katagiri-mx.com} | \href{https://www.linkedin.com/in/osamu-k-84b2b940}{linkedin}\\ \href{mailto:osamu.katagiri@exatec.tec.mx}{osamu.katagiri@exatec.tec.mx}\\
+52 55 2300 9176
}

\vfill

%%%%%%%%%%%%%%%%%%%%%%%%%%%%%%%%%%%%%%
%
%     COLUMN ONE
%
%%%%%%%%%%%%%%%%%%%%%%%%%%%%%%%%%%%%%%

\begin{minipage}[t]{0.34\textwidth} 

%%%%%%%%%%%%%%%%%%%%%%%%%%%%%%%%%%%%%%
%     EDUCATION
%%%%%%%%%%%%%%%%%%%%%%%%%%%%%%%%%%%%%%

\section{Education} 

\subsection{Tecnológico de Monterrey}
\descript{MSc in Nanotechnology}
\location{Jan 2019 - Dec 2020 | Estado de México, MX}
\location{Cum. GPA: 3.8 / 4.0}
\sectionsep

\subsection{Tecnológico de Monterrey}
\descript{BS in Digital Systems and Robotics}
\location{Aug 2012 - May 2016 | Querétaro, MX}
\location{Cum. GPA: 3.5 / 4.0}
\sectionsep

%%%%%%%%%%%%%%%%%%%%%%%%%%%%%%%%%%%%%%
%     LINKS
%%%%%%%%%%%%%%%%%%%%%%%%%%%%%%%%%%%%%%

\section{Links} 
Github:// \href{https://github.com/katagirimx}{\bf katagirimx} \\
LinkedIn:// \href{https://www.linkedin.com/in/osamu-k-84b2b940}{\bf Osamu Katagiri-Tanaka} \\
Personal Website:// \href{https://www.katagiri-mx.com/}{\bf katagiri-mx.com}

%%%%%%%%%%%%%%%%%%%%%%%%%%%%%%%%%%%%%%
%     COURSEWORK
%%%%%%%%%%%%%%%%%%%%%%%%%%%%%%%%%%%%%%

\section{Coursework}
\subsection{Graduate}
Nano-structured Materials \\
Carbon Nano-materials \\
Plastics and Composites Engineering \\
\textit{\textbf{Rheology \& Electrospinning}} \\
\sectionsep

\subsection{Undergraduate}
Sensors \\
Control Engineering \\
Digital Systems \\
Computer Architecture \\
Embedded Systems \\
Web Application Design \\
Microcontrollers \\
Electric Circuits \\

%%%%%%%%%%%%%%%%%%%%%%%%%%%%%%%%%%%%%%
%     SKILLS
%%%%%%%%%%%%%%%%%%%%%%%%%%%%%%%%%%%%%%

\section{Skills}
\subsection{Programming}
\location{Over 5000 lines:}
Python \textbullet{} Javascript \textbullet{} \LaTeX\ \\
\location{Over 2000 lines:}
C \textbullet{} C++ \textbullet{} ADA \textbullet{} Verilog \textbullet{} VHDL \\
\location{Over 1000 lines:}
Java \textbullet{} CSS \textbullet{} PHP \textbullet{} Assembly \\
\location{Familiar:}
Android \textbullet{} MySQL
\sectionsep

%%%%%%%%%%%%%%%%%%%%%%%%%%%%%%%%%%%%%%
%
%     COLUMN TWO
%
%%%%%%%%%%%%%%%%%%%%%%%%%%%%%%%%%%%%%%

\end{minipage} 
\hfill
\begin{minipage}[t]{0.65\textwidth} 

%%%%%%%%%%%%%%%%%%%%%%%%%%%%%%%%%%%%%%
%     EXPERIENCE
%%%%%%%%%%%%%%%%%%%%%%%%%%%%%%%%%%%%%%

\section{Experience}
\runsubsection{GE Aviation}
\descript{| Embedded Software Eng. }
\location{Jun 2018 - Dec 2018 | Querétaro, MX}
\vspace{\topsep} % Hacky fix for awkward extra vertical space
\begin{tightemize}
\item At General Electric's Business \& General Aviation (BGA) Power Software team, I develop and test critical software for Aviation Power products. I have high responsibility in the development and in the documentation of the features and interactions with other systems.
\end{tightemize}
\sectionsep

\runsubsection{GE Aviation}
\descript{| SW Edison Engineering Development Program }
\location{Jun 2016 – May 2018 | Querétaro, MX}
\begin{tightemize}
\item EEDP is an intensive program for people who have a passion for technology, a drive for technical excellence, and share in GE's core values. It is designed to accelerate participants' professional development through intense technical training.
\item 1st rotation at GEIQ-BGA software validation \& verification team
\item 2nd rotation at GEIQ-N\&G tools team
\item 3rd and 4th rotations at GEIQ-BGA software development team
\end{tightemize}
\sectionsep

\runsubsection{GE Power}
\descript{| Software EID Intern }
\location{May 2015 – May 2016 | Querétaro, MX}
\begin{tightemize}
\item Support and improve engineering projects and activities.
\item Worked on the analysis and optimization of +20 wind turbines for every GE wind farm worldwide.
\end{tightemize}
\sectionsep

%%%%%%%%%%%%%%%%%%%%%%%%%%%%%%%%%%%%%%
%     RESEARCH
%%%%%%%%%%%%%%%%%%%%%%%%%%%%%%%%%%%%%%

\section{Research}
\runsubsection{Macrophotoscience Reseach Group}
\descript{| MSc Student}
\location{Jan 2019 – Dec 2020 | Nuevo León, MX}
Worked with \textbf{\href{https://orcid.org/0000-0003-0455-5401}{Phd. Alan Aguirre}} and \textbf{\href{https://orcid.org/0000-0001-5325-0079}{Phd. Dora Medina}} to determine the electro-spinnability of various polymer solutions for the fabrication of carbon nano-wires.
\sectionsep

%%%%%%%%%%%%%%%%%%%%%%%%%%%%%%%%%%%%%%
%     AWARDS
%%%%%%%%%%%%%%%%%%%%%%%%%%%%%%%%%%%%%%

\section{Awards} 
\begin{tabular}{rll}
May 2018	& top 4\%                   & Software EEDP graduate at GE Aviation\\
Aug 2015	& 1\textsuperscript{st}/100 & GE 9th Lean Challenge\\
Nov 2014	& 1\textsuperscript{st}/50  & GEIQ’s Robotics Project\\
\end{tabular}
\sectionsep

%%%%%%%%%%%%%%%%%%%%%%%%%%%%%%%%%%%%%%
%     PUBLICATIONS
%%%%%%%%%%%%%%%%%%%%%%%%%%%%%%%%%%%%%%

\section{Publications} 
%\renewcommand\refname{\vskip -1.5cm} % Couldn't get this working from the .cls file
\vspace{\topsep} % Hacky fix for awkward extra vertical space
\vspace{\topsep} % Hacky fix for awkward extra vertical space
\bibliographystyle{abbrv}
\bibliography{publications}
\nocite{*}

\end{minipage}

\vfill

\end{document}  \documentclass[]{article}
