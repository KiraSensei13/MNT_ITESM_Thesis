% Chapter Template

\chapter{Methodology} % Main chapter title

\label{Chapter:Methodology}

%\subsubsection*{\color{mygray}[Chapter under work]}
% process or set of steps that are to take place in order to fulfill the objectives. These steps must mention the experiments to conduct, how they are going to be carried out, the evaluation of the obtained results, the validation of the hypothesis, the answers to the research questions, and the last step must be the written report of the outcomes

The following describes the proposed work to be done to fulfill the objectives stated in this document. The tasks are grouped in several work packages as described below:

%----------------------------------------------------------------------------------------
%	SECTION 1
%----------------------------------------------------------------------------------------

\section{Preliminary Literature Review}
The first step to the thesis development is the study of preliminary literature and related work. The purpose of this work package is the familiarization of the existing techniques such as: far and near field electrospinning, lithography, pyrolysis, carbonization and photopolymerization. On the other hand, some research is to be done in order to recognize if there are any efforts in the design of electrospun-able, photopolymerizable and pyrolysable polymer solutions.

Furthermore, the motive of this work package is to find common parameters that could link the techniques mentioned above for the fabrication of carbon nano-wires from polymer solutions that can be electrospun by NFES, photopolymerized and then pyrolyzed. This work package is to be carried out through the entire thesis development process, as the state-of-the-art may change within that period of time.

\section{Evaluation of Fabrication Parameters}
As the polymer solution is the principal input to the proposed technique (See Figure \ref{fig:fabricationFlowChart}), it is required to identify and understand the fabrication parameters that have an impact on the quality of the carbon nano-wires. For that reason two tasks are to be executed:

\begin{itemize}
	\item Study and identify the process parameters that influence the fabrication of carbon nano-wires
	\item Study and identify the rheological properties in polymer solutions that affect the electrospinning and pyrolysis techniques
\end{itemize}

\section{Polymer Solution Design}
Once the process parameters and rheological properties that affect the fabrication of carbon nano-wires are identified, the design process shall take place. This work package is to study polymer solutions that can be electrospun by NFES, photopolymerized and pyrolyzed. The polymer solution design will comprise of two steps:

\begin{itemize}
	\item Prepare and test various polymer solutions with specific distinctions according to the identified solution properties and process parameters.
	\item Perform rheological analyses to determine if the prepared polymer solutions can be employed for the fabrication of carbon nano-wires.
\end{itemize}

\section{Fabrication of Carbon Nano-wires}
From the rheological analyses, determine and control the polymer solution properties and fabricate carbon nano-wires.
	
This work package intends to involve several manufacturing processes (near field electrospinning, photopolymerization, pyrolization and carbonization) for the fabrication of carbon nano-wires. This task will require the integration of several techniques:

\begin{itemize}
	\item \emph{Electrospinning} - to convert the polymer solution into polymer nano-fibers
	\item \emph{Photopolymerization} - to change the chemical properties of the polymer solution and crosslink its molecules. This is to prevent the polymer to melt during pyrolysis \cite{Basu2018}.
	\item \emph{Pyrolysis} - to transform the polymer nano-fibers into conductive carbon nano-wires.
\end{itemize}

See Figure \ref{fig:fabricationFlowChart}.
	
\section{Data Collection and Analysis of Results}
The data collection work package comprehends the study of the created carbon nano-wires using the new design polymer solution. The purpose is characterize the carbon nano-wires and compare them to the carbon nano-structures produced by existing techniques.

\section{Documentation}
Finally the documentation refers to the Thesis writing tasks. This task is intended to carry out through the entire thesis development process, as every work package above is to be referenced within the thesis document.

%-----------------------------------
%	SUBSECTION 1
%-----------------------------------


